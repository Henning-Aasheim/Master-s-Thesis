\lettrine{W}{hat came to light in the preceding chapter} needs some context and explanation. The theories developed against the backdrop of pre-existing research was wrong in regard to China's impact on its partners, and I need to explain why this is the case. Is this a limited area where the theory does not fit? or does this extend to autocratic diffusion theory in general?

I start this chapter with a discussion of what I found and how this relates to my research question, which is: \textit{Can linkages to major powers affect freedom of expression in other countries?} I then go on to show how my results impact the wider study of democratic backsliding, before giving a brief overview of the limitations and strengths of my research. 

As I find that linkages to China do not affect freedom of expression, and this is robust to several specifications of the model. I conclude that my study resulted in a null-finding, which might not be that interesting, however, it is important. It lets us know that it is very unlikely that China, and indeed any large autocratic country, can have a large negative effect on freedom of expression, and this can guide us in creating new hypotheses for studying international factors in democratic backsliding. 

\section{Evaluating the Models}
In the first section of this chapter, I evaluate the four tables presented in the analysis chapter. My results show no evidence that increased linkages to China have a negative impact on freedom of expression. While all of my models show a general negative trend, the effect sizes are small and the uncertainty around the estimates are high. These findings suggest that we should reject the two hypotheses posited in Section \ref{sec:hypotheses}.

\subsection{Relationship between linkages to China and freedom of expression}
The first expectation I had was that more linkages to China would cause freedom of expression to decline in other states. From the results of my models, this seems not to be the case. In all my models  I get a non-significant result when regressing linkages to China on freedom of expression. This indicates that when controlling for other influencing processes, more linkages to China are not associated with less freedom of expression. The signs are negative in all cases, but the standard errors are too large to even be sure about the direction of the true effect. In addition to this, the effect size is negligible, meaning that even if there was an effect, this should not be considered a major explanatory factor. Given this result, we should still assume the null hypothesis that linkages to China do not have an effect on freedom of expression. I will consider this first, then extend the discussion to include other reasons. 

Earlier in the literature chapter I discussed previous findings concluding that China is not actively attempting to spread its form of governance. China's neutral stand on autocratisation gives additional cause to keep the null-hypothesis. My theory, however, was that domestic players might use China to restrict freedom of expression, which it does not seem like they do. So why is this? First, while China's influence has grown considerably over the past three decades, the growth has been quite even, with few countries seen large changes in their ties. The slow growth of linkages, rather than happening all at once, might also dull the impact it has on freedom of expression, should there happen to be one. Second, the Chinese model's popularity is questionable. China's system might be an ideal for some leaders, but the size of China might make it difficult to learn from. As state capacity in most autocratising regimes is too weak, implementing restrictive measures learned from China might not be feasible. Additionally, China's growth might have replaced linkages to the West in some regions of the world, but this displacement is not happening evenly, with many countries actually becoming even more strongly connected with the West in the period. Thus, many of the theories of why China should be able to influence freedom of expression in other countries might be valid, but China might effectively be unable to assert its influence strongly enough to have an impact.

In addition to this, many of the contributions, while not outright rejecting that international factors plays a role, seem to advocate a more cautious approach to studying diffusion and putting more emphasis on domestic factors \citep{bader_china_2015, buzogany_illiberal_2017, borzel_noble_2015, risse_democracy_2015}. As I do not find evidence of linkages to China influencing freedom of expression -- which after all is the component of the democracy score which has seen the largest drop in the last ten years -- I consider this to be probable. 

It is also hard to discount the possibility that it is not China in isolation, but rather authoritarian countries together that causes freedom of expression in partner countries to decline. While this has not been the focus of this study, calculating the total influence of of some of the major autocracies, like China, Russia, Iran, and Saudi Arabia might give us another valuable perspective on how linkages can facilitate decrease in freedom of expression. This study can be thought of as a most likely case for the influence of linkages to an authoritarian country and freedom of expression, the result of the study going a long way to disproving the theory. However, I have not explored how including several of the largest Black Knight countries effect the outcome. 

While I think the most plausible explanation is that linkages to China have no effect on freedom of expression, there is always a chance that my models are misspecified in some way, or that the measures of my concepts do not hold up. I discuss limitations of my concepts below; however, I do not believe my models to be severely misspecified. I have run several different models with fixed effects and control variables to limit the possibility of omitted variable bias, making me confident in them. While some previous studies have found that linkages to China do have an impact on media freedom and self-censorship \citep{gamso_is_2021, toettoe_foreign_2023}, I believe that when aggregated into the Freedom of Expression index, this relationship disappears. 

\subsection{Relationship between linkages to China and freedom of expression by regime type}
My second expectation was that linkages to China would have a greater negative impact on freedom of expression in hybrid regimes. This was an expectation derived from previous studies, which have mainly focused on hybrid regimes \citep{tansey_ties_2017, toettoe_foreign_2023}. This makes sense as hybrid regimes are more malleable and would likely be more prone to outside influence. At the same time, hybrid regimes can both improve and worsen civil liberties, and it is a rather heterogenous group of countries. This can serve to make possible a situation where linkages to China can both strengthen and reduce freedom of expression for different countries, obscuring any effects that might be there.

I tested my expectation by interacting linkages to China with a regime type variable of four categories: closed autocracies, electoral autocracies, electoral democracies, and liberal democracies (Table \ref{tab:h2}). None of the results are significant, nor are they sizeable, and the differences between the regime types are small. From Table \ref{tab:h2} it actually looks like hybrid regimes are less influenced by linkages to China. I tested this for several different leads as well, and here I found a very surprising result. When using a ten-year lead on the dependent variable I find a significant and sizeable result for the hybrid regimes. This is very surprising, especially as the coefficient of the interaction term between linkages to China and electoral democracy is estimated to -0.667, which, while not too large, is large enough to have an impact.

Model 2.10 which shows the significant result is interesting, however, even if it is significant, I would caution against reading too much into it. The reason is that ten years is a lot of time. According to this model, an increase in linkages to China in a certain year is estimated to produce a large effect some ten years later. This is not a credible assertion, as I believe there are too many variables at play in that time frame for the model to properly estimate the relationship. It is more likely that by choosing a lead of ten years I have accidentally created an artificial and spurious relationship between linkages to China and freedom of expression. When running a lot of models to test the results, this is a danger that we should be conscious of. Thus, even if I have a significant result, I do find it to be an unlikely one. 

I want to discuss some of the reasons why hypothesis two seems to be disconfirmed. As always there are one of two main reasons.  Either there is no difference in the relationship between linkages to China and freedom of expression for hybrid  and institutionalised regimes, or the relationship is somehow obfuscated. The first explanation I have is that there just is no difference in how linkages to China impact countries with different regime types. Since none of the results are significant, the cautious approach would be to keep the null hypothesis of no difference. The errors of the estimates are too large for me to say anything definite, but it should nevertheless be accepted that it is unlikely that linkages to China have different effects on hybrid versus institutionalised regimes. Why would this be the case? The likely answer would be that since linkages to China does not have an effect on freedom of expression, it does not matter what type of regime the partner country has. Before running the second model, it was possible that a significant and substantial find was obscured by differences between regime types. However, since this is not what I found, we actually increase our confidence that hypothesis one was false.

The second reason hypothesis two had no significant results might be that, as explained above, the models struggle with estimating the coefficients for hybrid regimes. This is not that unlikely, as we can see in the residual plot (Figure \ref{fig:residuals}). Hybrid regimes might have different actors pushing and pulling in different directions, making it exceedingly hard for my model to correctly measure the effect linkages to China has on freedom of expression. In the residual plot, the predictions in the centre are less well estimated than at either side of the plot. If we consider it more likely that most closed autocracies have almost no freedom of expression, while most liberal democracies have almost full freedom of expression, which they have almost by definition, we should find the centre of the plot almost exclusively occupied by hybrid regimes. It is here that the biggest outliers can be found, some countries experiencing almost revolutionary changes. We should not expect our model to be able to predict these observances correctly, but since they are included when running the model, this might cause problems when estimating the coefficients for these observations. Removing these observations might seem to be a way to solve this, but they are part of the data, and we should be cautious of removing observations only because they do not fit with our expectations. I would rather take this uncertainty to mean that linkages to China are less important, than removing them. This measurement problem is the reason I am less confident in rejecting hypothesis two. However, because of the small coefficient sizes being estimated for all regime types, I conclude that there is an overwhelming probability that hypothesis two is false. 

\subsection{Linkages to the West}
I briefly need to discuss an additional find.  In many of my models I found that linkages to the West was negatively associated with freedom of expression and statistically significant. Thus, more linkages to the West are predicted to decrease the freedom of expression score.

Investigating the size of the coefficients in Section \ref{sec:west_coefficients}, I find that with large differences in ties, this might have some impact on freedom of expression scores. This is peculiar, because I had assumed the opposite. Western regimes explicitly states democracy promotion as a goal of their foreign policy (\citeauthor{borzel_noble_2015} \citeyear{borzel_noble_2015}, p. 523; \citeauthor{levitsky_linkage_2006} \citeyear{levitsky_linkage_2006}). While this made me expect a positive relationship, this is not the case. There is more evidence for this, with \citet[pp. 523-524]{borzel_noble_2015}, \citet[pp. 464-466]{delcour_spoiler_2015} \citet{freyburg_local_2015} finding that Western countries sometimes prioritise security and stability over democracy promotion. \citet{delcour_spoiler_2015} further finds evidence that when authoritarian states engage in autocracy promotion, it can backfire. It would not be unlikely if this was to happen with democracy promotion as well, especially without domestic drivers of democratisation \citep{risse_democracy_2015}.

I consider the estimations of the linkages to the West to be plausible, however, my model is not specified to examine this relationship. So, while the results should be taken seriously, the reader should keep in mind that this relationship needs more research.

\section{Impacts of the Results on the Wider Discussion}
Finding that linkages to China do not impact freedom of expression, I want to consider what this result implies for the wider discussion around linkages and democratic backsliding. The literature is full of contradictory evidence for the impact of linkages, some arguing for \citep{gamso_is_2021, loughlin_chinese_2021, toettoe_foreign_2023} and some against \citep{brownlee_limited_2017, wong_chinese_2019, yilmaz_authoritarian_2020}. The reason seems to be that there is both evidence that confirm and contradict the impact of linkages on autocratisation, especially when looking at individual countries \citep{loughlin_chinese_2021, yilmaz_authoritarian_2020, wong_chinese_2019}. 

Quantitative studies show similar tendencies, with \citet{gamso_is_2021} and \citet{toettoe_foreign_2023} finding significant results from linkages to China, while \citep{bader_china_2015} does not. The difference here being the dependent variable. Where \citet{bader_china_2015} measures China's effect on democracy (at least the stability of it), \citet{gamso_is_2021} and \citet{toettoe_foreign_2023} measure the impact of linkages to China on the media. The former is a highly aggregated measure, where China might have both positive and negative effect depending on the component measured. The latter, on the other hand, are less aggregated, with the effect likely being less influenced by other variables working in the opposite direction.

This study has attempted to straddle a middle way between these two, focusing on freedom of expression, a defining feature of democracy that has been rapidly declining in recent years. The results is that linkages to China are not likely to be seriously affecting freedom of expression in other countries. This is likely down to the fact that China's influence is non-existent or too small to be measured. While both hypotheses were wrong, this is nonetheless an interesting result. The discourse around China usually sees it as a force of autocratisation \citep{jintao_chinas_2023, biden_remarks_2021, economy_exporting_2020, repucci_authoritarians_2022}, however, this study helps to corroborate massing evidence that China's influence on other countries is far smaller than first believed \citep{saba_promoting_2025, borzel_noble_2015, risse_democracy_2015, hackenesch_not_2015}, at least in isolation.

In the results I find further evidence for the fact that Western countries might not care all that much about democracy and freedom of expression in the face of more authoritarian competition. The coefficient for the linkages to the West is almost always negative, and usually statistically significant. It might not be large, meaning there is little evidence that it is driving a substantial decrease in freedom of expression. However, I would argue that it is not too unlikely that domestic actors in autocratising countries are using linkages to the West to keep themselves relevant. This would mean that the immunity mechanism which I proposed was a way for China to leverage its linkages is less important than I had first assumed. China might be an alternative, but it does seem like Western countries are fine with supporting autocratisers as long as stability and security is ensured. My assumption was that this was the strongest mechanism, however, with this result, linkages to China and any other authoritarian country seem to matter less for autocratisation than has so far been argued. This is an important find, because it strengthen arguments that Western countries are less keen on democratisation than the academic and public discourse suggest \citep{borzel_noble_2015, delcour_spoiler_2015, freyburg_local_2015, risse_democracy_2015}.

This has implication for future research. First and foremost, it would perhaps be interesting to look more into international factors as they meet with their domestic counterparts. It might be in the confluence between domestic and international factors where foreign entities might have the largest effect. This might require more qualitative enquiry; tracing how authoritarian countries use their foreign relations prowess to influence domestic affairs in partner countries. Using linkages to influence other countries might work when China actually pushes for it, but this is likely to be a limited endeavour, which will not show up in quantitative research. 

Second, focusing on China in isolation is likely not a fruitful approach, unless there is some very good reason for doing so. Using the FBIC dataset \citep{moyer_china-us_2021}, linkages to China have rather low variation, which makes estimation hard. A more interesting option would be to focus on several of the larger autocracies (China, Russia, Iran, Saudi Arabia, and Venezuela) and see how they, in aggregate, affect either freedom of expression specifically, or democracy in general. This would allow for more variation in the data, which makes would make it easier to model the relationship. This is somewhat similar to the approach of \citet{tansey_ties_2017}, just shifting the independent variable to focus on the Black Knights, instead of excluding them. When doing this, it would also be interesting to dive more into the mechanisms and how they work. Looking at learning could, for instance, be done by studying the effect of diplomatic linkages. Immunity could be studied with more focus on economic linkages. And finally, displacement could be better studied by looking at the differences in linkages, as opposed to just focusing on their total size.

While above I stated that focusing on China in isolation was likely to be a mistaken approach, I will concede that there are still ways of doing so. One avenue of research I find particularly interesting is to estimate the effect of linkages to China on internet censorship. This variable is not included as a part of the V-Dem's Freedom of Expression index, but because of China's advanced internet censorship capability, it is not unlikely that linkages to China might impact this variable. It is also possible that China is able to reshape the norms of the international system. This is not something linkages are able to fully discover, as norms can spread even without linkages. This is also something future research can look into, because China's success in the last decades lacks historical precedence.

\section{Limitations and Strengths}
In this section I will discuss some of the limitations and strengths of this study. The world is complex, and we need to be aware of what might bias the results. There are both conceptual and methodological limitations to be conscious of, but at same time I have designed my research in such a way as to limit threats to inference. I try to be fully transparent about how I run my models and actively try to check the validity and robustness of the results. I start by discussing limitations before moving on to the strengths of my project.

\subsection{Limitations}
The first limitation to be countenanced is the possibility that I have neglected to include an important explanatory variable. The world is complex, and the likelihood of omitting an important variable is always higher than any quantitative researcher would like to admit. By using two-way fixed effects model controlling for country and year, I have tried to preclude factors that are constant over time and for each individual country, reducing the risk of omitted variable bias. In addition to this, I have included several other possible variables that are likely to vary both between countries and over time. To ensure that these are viable to use with my framework, I have tried to pick out variables that are shared between other recent studies using the same modelling approach \citep{gamso_is_2021, toettoe_foreign_2023}. All these procedures makes the risk of bias smaller, but a good deal of the variation between the observations are removed, and my model should be seen as a conservative one. 

The second problem is whether or not my variables are able to correctly tap into the concepts. The dependent variable is gathered from the V-Dem index, with many qualified experts contributing to it. However, at its core it is a subjective index, and this might not be able to properly measure what we actually mean by freedom of expression, and some of the differences might be artificial. Even with this possibility, I still consider the Freedom of Expression and Alternative Sources of Information index to be the best operationalisation of my dependent variable.

While I believe V-Dem's freedom of expression index to be the best possible conceptualisation of my dependent variable, it does lack one significant feature, which is internet censorship. This variable is included in the full V-Dem dataset; however, it is not included in the Freedom of Expression and Alternative Sources of Information index. I did want to include it, but because of time constraints I was not able to do this. The reason for including it is that China is especially proficient at internet censorship, however, this might be an interesting avenue for future research.

A bigger source of problem might lie in the FBIC-index. This index, while being close, might not exactly measure what I am interested in. Influence and linkages are similar, but not equal, and the index excludes some of the linkages proposed by \citet{levitsky_competitive_2010}.\footnote{See Table \ref{tab:dimensions} on page \pageref{tab:dimensions}.} It includes trade, aid, security, diplomacy, etc., yet it fails to capture social linkages, which might be an important, yet unobservable variable. Social ties might affect both linkages to China and freedom of expression which could cause omitted variable bias, however, this is unlikely because of the two-way fixed effects model I use. Another problem might lie in the fact that \citep{moyer_china-us_2021} considers all linkages in intergovernmental organisations to be political linkages. And, while this is true to an extent, China-dominated intergovernmental organisations are far more likely to have an effect. Almost all countries participate in the UN, only some in the Shanghai Cooperation Organisation. The linkage variable for China also does not have that much variation, and this is a problem in any regression analysis, but even more so for a fixed effects one. It could have been preferable to make a new, more varied, and more precise dataset. However, this would not be feasible on this short a time frame. This should, however, be considered if redoing the analysis, or when looking into other phenomenon where linkages to China is hypothesised to play a part.

\subsection{Strengths}
While there are some conceptual limitations, I consider my research design to be a strong one. By using country and time fixed effects and including several control variables, I try to limit the threat of omitted variable bias which impacts all observational studies. Fixed effects reduces bias because it excludes much of the variation that is created by other variables. This makes the estimates conform more to the real world. However, there is a backside to this procedure, and that is that variance is removed. With less variance, the errors in the estimates gets larger, and this reduces confidence in them. My model is a conservative one; being less likely to produce a result, but it protects against bias. 

Another way the use of fixed effects is useful, is that it controls for many possible confounders without including too many control variables. This makes the model less complicated to work with, and safe-guards against the risk of opening unintended backdoor paths when including many control variables \citep[Chapter 3]{cunningham_causal_2021}.

I also try to describe the results in detail, making the coefficients easy to interpret. This is important when using fixed effects, as the coefficients are slightly harder to understand than in a normal regression model. Using examples is also a good way of showing the actual impact of the estimated effect. Just having a statistically significant result is not enough to confirm a result and the actual impact must be non-negligible to assert a relationship. 

Additionally, I run several models with different specifications (Appendix \ref{apn:robust}) where I test how sensitive my results are to changes. I find that, while there are smaller differences, on the whole my models are stable, increasing my trust in the estimations. 

\section{Summary}
In this chapter I have discussed the results of my models and how they behave in the face of additional specifications. I find no support for my two hypotheses. I consider it very likely that linkages to China are not able to explain decrease, or any variation, in freedom of expression. First of all, China has no stated mission of forcing autocratisation in other countries, rather wanting to be a neutral partner. Second, is that one of the major mechanisms I proposed, immunity, seems to be less relevant than expected. Linkages to Western countries were consistently negative, but only weakly so, possibly indicating that Western countries care less about democracy and freedom of expression in their foreign policy, and more about stability and security. This was tentatively indicated in the literature, but my findings give some additional support for this theory.

Most likely because there is no relationship between linkages to China and freedom of expression, I found no evidence for my second hypothesis. I hypothesised that hybrid regimes would be more affected by linkages to China than would an institutionalised regime, but no such evidence emerged. I was, however, more reluctant to conclude with a null find on the second hypothesis, because I had less data. However, as the relationship was estimated to be very small, I have ended up preferring the null-hypothesis. This, in turn, strengthens my belief that hypothesis one was wrong, since I now know that the difference between regime types was not obscuring the effect of the Chinese linkages.