\lettrine{W}{hat came to light in the preceding chapter} needs some context and explanation. The theories developed against the backdrop of pre-existing research was wrong in regards to China's impact on its partners, and we need to understand why this is the case. Is this a limited area where the theory does not fit, or does this extend to autocratic diffusion theory in general?

I start this chapter with a discussion on how the findings of the analysis might give an answer to my research question, which is: \textit{can major powers affect freedom of expression in other countries?} I have no clear answer, but the most likely one based on this study is: possibly, but probably not. This is not a clear answer, but there are unfortunately too many variables in play to be able to say anything for certain. This should be the main takeaway from this study, however, I will now discuss this in greater depth.

\section{Evaluating the Models}

\subsection{Direct relationship between linkages to China and freedom of expression}
The first expectation I had was that more linkages to China would cause freedom of expression to decline in other states. From the results of my models, this seems not to be the case. In my main models, and most of the robustness models, I get a non-significant result when directly regressing linkages on freedom of expression. This indicates that controlling for other influencing process, more linkages to China is not associated with less freedom of expression. The sign of the result also fluctuates quite a lot with different specifications, strengthening this assessment.

This result stands in sharp contrast to my first expectation. Why is this? There are many possible explanation, with the most important being the possibility that linkages to China do not have an effect on freedom of expression. I will consider this first, then extending the discussion to include other reasons. In the literature chapter I discussed earlier findings concluding that China is not actively attempting to spread its form of governance. With this we can be rather assured that China is not pushing for restrictions on freedom of expression. Many of the contributions, while not rejecting that international factors may play a role, also seem to advocate a more cautious approach and putting more importance on domestic factors \citep{bader_china_2015, }

\subsection{Relationship between changes in linkages to China and freedom of expression}

Some results that should be noted is that in Table \ref{tab:h1_delta} the linkage variable was always positively associated with freedom of expression and in some models was even significant at the ten per cent level. 

\subsection{Direct relationship between linkages to China and freedom of expression by regime type}

\subsection{Relationship between changes in linkages to China and freedom of expression by regime type}

\section{Limitations and Strengths}