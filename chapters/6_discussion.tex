\lettrine{W}{hat came to light in the preceding chapter} needs some context and explanation. The theories developed against the backdrop of pre-existing research was only partially right in regards to China's impact on its partners, and we need to understand why this is the case. Is this a limited area where the theory does not fit, or does this extend to autocratic diffusion theory in general?

I start this chapter with a discussion on how the findings of the analysis might give an answer to my research question, which is: \textit{can major powers affect freedom of expression in other countries?} I have no clear answer, but the most likely one based on this study is: yes, but we need to factor in that the results are different for different regime types and the effect seems to be rather small.  

\section{Evaluating the Models}
In the first section of this chapter I evaluate the four main model specification I ran in the last chapter. The findings are interesting, but unfortunately they are not very clear, so a thorough discussion of the results is in its place. I will spend quite some time discussing Table \ref{tab:h2_delta}, as this is the most interesting in light of my theoretical assumptions; however, the null findings of the other tables deserves a discussion as well, as they leave more to be explored in the future.

\subsection{Direct relationship between linkages to China and freedom of expression}
The first expectation I had was that more linkages to China would cause freedom of expression to decline in other states. From the results of my models, this seems not to be the case. In my main models, and most of the robustness models, I get a non-significant result when directly regressing linkages to China on freedom of expression. This indicates that when controlling for other influencing processes, more linkages to China is not associated with less freedom of expression. The sign is negative in all cases, but the standard error is large enough that we cannot be sure that this is the case

While the result gives some support to my theory, the error margins are too big to conclude anything, and we should still assume the null-hypothesis that linkages to China do do not have an effect on freedom of expression. I will consider this first, then extend the discussion to include other reasons. 

In the literature chapter I discussed earlier findings concluding that China is not actively attempting to spread its form of governance. With this we can be rather assured that China is not pushing for restrictions on freedom of expression. Many of the contributions, while not rejecting that international factors may play a role, also seem to advocate a more cautious approach and putting more importance on domestic factors \citep{bader_china_2015, buzogany_illiberal_2017, risse_democracy_2015}. On the surface, I would argue that the findings from this study support this position. The literature in question is also mainly concerned with autocratisation, of which freedom of expression is but one part. Thus, even if we find no relationship between more linkages to China and freedom of expression, China might still be able to influence autocratisation through different pathways. We also cannot discount the possibility that it is not China in Isolation, but rather authoritarian countries together that causes freedom of expression in partner countries to decline. While this has not been the focus of this study, calculating the total influence of of some of the major autocracies, like China, Russia, Iran, and Saudi Arabia might give us another valuable perspective as to how linkages can facilitate decrease in freedom of expression. 

While, in light of the results in Table \ref{tab:h2_delta}, I consider it relatively unlikely that linkages to China have strictly negative effects on freedom of expression, we should still leave this door open. There is always a chance that my models are misspecified in some way, or that the measures of my concept does not hold up. In addition, other studies have found that linkages to China does have an impact on media freedom and self-censorship \citep{gamso_is_2021, toettoe_foreign_2023}, making it possible that even at the aggregation level of an index such as the V-Dem's Freedom of Expression and Alternative Sources of Information index, this connection disappears. 

\subsection{Relationship between changes in linkages to China and freedom of expression}
As an addition to my first expectation, I wondered whether changes in linkages to China over a sizeable span of time would have an effect on freedom of expression. This I examined in Table \ref{tab:h1_delta}, where the linkage variable was always positively associated with freedom of expression, and in some models was even significant at the ten per cent level. This is compounded by the fact that using the bandwidth variable made the results statistically significant at the five percent level. This result is by far the most challenging result to interpret, and fly straight in the face of my expectations. On what basis can we explain why an increase in linkages to China, even when accounting for most other influencing factors, might strengthen freedom of expression?

A possible explanation is that countries with an open economy, even if it runs well or not, is more likely to embrace freedom of expression. However, this possibility is made less likely by the find that the coefficient of linkages to the West is negative. It should also be remembered that it is change in linkages to China which sees this effect, not the absolute level of the linkages. Meaning that a country which had a growth in linkages the last three years is likely to see a strengthening of freedom of expression, even when controlling for increase in GDP per capita. The confluence of linkages to China being positive and linkages to the West being negative, might come down to the fact that linkages to the West is measured as absolute linkages, which declines when more open and free economies establish more and more ties to China leading to more open economies and pressure to liberalise and strengthen freedom of expression. This is one theory, but it is hard to confirm without going more into the details. This might make for several fruitful qualitative studies, going into the weeds of China's influence on its partners.

Another explanation, proposed in the literature, is that autocratisation attempts might cause a backlash, and actually help the cause for freedom and democracy \citep{delcour_spoiler_2015, risse_democracy_2015, way_limits_2015}. It might be the case that when countries establish linkages to China, it heightens the focus on the problematic nature of China's way of governing, and leads to an increase in freedom of expression. This might also help explain why using the bandwidth variable makes the result significant, because it is the number of linkages, not their size that stimulates discussion around the linkages to China.

These are just theories, and we cannot forget the fact that the results are inconclusive and may very well just point to the fact that there is no relationship between an increase in linkages to China and freedom of expression. There is also the possibility that the actual effect is obscured by the increase in linkages having different effect on different types of regimes, which will be discussed below. 

\subsection{Direct relationship between linkages to China and freedom of expression by regime type}
My second expectation was that linkages to China would have a greater negative impact on freedom of expression in hybrid regimes. This was an expectation derived from previous studies, which have mainly focused on hybrid regimes. This makes sense as hybrid regimes are more malleable and would likely be more prone to outside influence. In contrast to this, hybrid regimes can both improve and worsen civil liberties, and it is a rather heterogenous group of countries, where the linkages to China can both strengthen and reduce freedom of expression at the same time, obscuring any effects that might be there.

I first tested my expectation by regressing the absolute linkage score on freedom of expression (Table \ref{tab:h2}). I find that freedom of expression cannot be explained by linkages to China, and I also find evidence that hybrid regimes are less influenced by linkages to China than more consolidated regimes. This is contrary to my expectation and we need to know why. I have two main hypotheses for why my expectations were not met.

The first hypothesis is that there just is no difference in how linkages to China impact countries with different regime. Since none of the results are significant, the cautious approach would be to keep the null-hypothesis of no difference. The errors of the estimate are too large for us to say anything definite, but we should nevertheless accept that there is a possibility that more linkages to China does not cause there to be less freedom of expression.

The second hypothesis is that, as explained above, the model struggles with estimating the coefficients for hybrid regimes. Hybrid regimes might have different actors pushing and pulling in different directions, making it exceedingly hard for my model to correctly measure the effect linkages to China has on freedom of expression. I consider this to be plausible because of the results from the coefficient plot (Figure \ref{fig:residuals}). In the plot, the predictions in the centre are less well estimated than at either side of the plot. If we consider it more likely that closed autocracies have almost no freedom of expression, while liberal democracies have almost full freedom of expression, which they are almost by definition, we should find hybrid regimes to occupy the centre of the plot. It is here we find our biggest outliers, some countries experiencing almost revolutionary changes. We should not expect our model to be able to predict these observances correctly, but since they are included when running the model, this might cause problems estimating the coefficients for these observations correctly. 

Based on these hypotheses, we should be very careful of concluding too strongly for or against.

Looking a little closer at the results, keeping in mind the large errors, the effect sizes are actually the reverse of my expectation. This is to say that hybrid regimes are, according to the available data, less affected than more consolidated regimes. If choosing to believe the first hypothesis, that there are no relationships, this result does not mean much. With the second hypothesis, accepting however that the relationship is unbiased, which they should be, they would seem to suggest that freedom of expression in hybrid regimes is less effected by linkages to China than consolidated regimes. If this is the case, how can we explain this? It does seem unlikely that this should be the case, but I do not have any definitive answers for why this is the case, other than pointing out the likelihood that my results are coloured by the fact that hybrid regimes are more likely to have unexpected or extreme values, and that this in turn poses problems when estimating.

\subsection{Relationship between changes in linkages to China and freedom of expression by regime type}
The last table of chapter \ref{chp:analysis} includes the results from Equation \ref{equ:h2_delta}, looking at how changes in linkages to China over a three-year period impacts freedom of expression. This is the only table which gives me significant results and we should thus take a thorough look into what this means for both my first and second hypothesis.

First of all, the results of Table \ref{tab:h2_delta} seems to disprove hypothesis two, that linkages to China affects hybrid regimes more than consolidated ones. Indeed, liberal democracies seems to be slightly more effected by linkages to China than electoral democracies. Combined with the results of Table \ref{tab:h2}, this result  makes it very likely that there are no differences in how linkages affects hybrid and consolidated regimes. This is surprising as the theory would suggest that hybrid regimes would be more affected \citep{tansey_ties_2017, toettoe_foreign_2023}. As I write above, this might in the end come down to the fact that my models struggles with estimating coefficients for hybrid regimes, however, it is not impossible that linkages to China for some reason has a bigger impact on consolidated regimes. Maybe linkages to China trace an ebb-and-flow pattern for hybrid regimes, while for consolidated regimes the pattern is more consistent? This is a possibility to note and a possible avenue for future research.

This result also has an impact on my hypothesis one, where it opens the possibility that an increase in linkages to China can effect freedom of expression. However, the effect is more nuanced than I had initially expected. My expectation was that more linkages to China---or, indeed, an increase in linkages to China---would cause freedom of expression to decline. First I only find a relationship when using changes in linkages, and second, it seems that this is only the case for democratic regimes. For liberal and electoral democracies linkages to China has a negative effect on freedom of expression. So far this result accord somewhat with my expectation, however, closed autocracies actually saw linkages to China have a positive effect on freedom of expression. I cannot explain why an increase in linkages to China works positively on freedom of expression in closed autocracy. One reason for measuring a positive relationship is because of the floor effect, where it is almost impossible for closed autocracies to get a lower score. This, because of a general trend of increasing freedom of expression in the last thirty years is then estimated as a positive relationship. This, however, is nothing more than a hypothesis, derived from the results.

What all these results mean, is that my first expectation is conditionally confirmed. Of course, there is no evidence for a strictly negative relationship, but there is evidence that linkages to China impacts freedom of expression, with different effects depending on regime type. An interesting study would be to look more into hybrid regimes to try to tease out domestic processes which might combine with linkages to China to produce certain types of effects. With a general trend of increasing democratic backsliding, we need to know more about the regimes affected. This result might, with more study, shed some necessary light on this issue.

\section{Limitations and Strengths}
In this section I will discuss some of the limitations and strengths of this study. The world is complex and we need to be aware of what might bias the results. At the same time I designed my research in such a way as to limit threats to inference. However, perfection is impossible, because, as in the famous quote by Herodotus: `were it your aim always give equal weight to each and every circumstance as it impinged upon you, then you would never get anything done.' These words are as true today as they were 2500 years ago, and at some point we need to come up with an answer that is as right and true as possible given our limitations. I start by discussing limitations before moving on to the strengths of my project.

\subsection{Limitations}
The first we have to countenance the possibility that I have neglected to include an important explanatory variable. The world is complex and the possibility of omitted variable bias is always present. By using country and year fixed-effects, I have tried to preclude factors that are constant over time and for each individual country. In addition to this, I have included several other possible variables that are likely to vary over both between countries and over time. To ensure that these are viable to use with my framework, I have tried to pick out variables that are shared between other recent studies using the same modelling approach \citep{gamso_is_2021, toettoe_foreign_2023}.

The second problem is whether or not our variables are able to correctly tap into my concepts. The dependent variable is gathered from the V-Dem index, with many qualified experts contributing to it. However, at its core it is a subjective index, and this might not be able to properly measure what we actually mean by freedom of expression, and some of the differences might be artificial. Even with this possibility, I still consider the Freedom of Expression and Alternative Sources of Information index to be the best operationalisation of my dependent variable.

While I believe V-Dem's freedom of expression index to be the best possible conceptualisation of my dependent variable, it does lack one significant feature, which is internet censorship. This variable is included in the full V-Dem dataset, however, it is not included in the Freedom of Expression and Alternative Sources of Information index. I did want to include it, however, because of time constraints I was not able to do this. But, since China is especially proficient at internet censorship, it might be interesting avenue for future research.

A bigger source of problem might lie in the FBIC-index. This index, while being close, might not exactly measure what I am interested in. Influence and linkages are similar, but not equal, and the index excludes some of the linkages proposed by \citet{levitsky_competitive_2010}.\footnote{See Table \ref{tab:dimensions} on page xx.} It includes trade, aid, security, diplomacy, etc., yet it fails to capture social linkages, which might be an important, yet unobservable variable. Social ties might indeed affect both linkages to China and freedom of expression which could cause omitted variable bias. Another problem might lie in the fact that \citep{moyer_china-us_2021} considers all linkages in intergovernmental organisations to be political linkages. And, while this is true to an extent, China-dominated intergovernmental organisations are far more likely to have an effect. Almost all countries participate in the UN, only some in the Shanghai Cooperation Organisation. 

\subsection{Strengths}
While there are some conceptual limitations, I consider my research design to be quite strong. By using country and time fixed-effects and including several control variables, I try to limit the threat of omitted variable bias which impacts all observational studies. The perfect design might be an experiment, but this being impossible, my design is the next best thing.

Additionally I run several models with different specifications (Appendix \ref{apn:robust}) where I test how sensitive my results are to changes. I find that while there are smaller differences, on the whole my models are stable, increasing my trust in the estimations.

\section{Summary}
In this chapter I have discussed the results of my models and how they behave in the face of additional specifications. I find that the first hypothesis, in the strict sense is disconfirmed, but contingently confirmed when looking at the impacts of linkages to China on different regime types. I find that these results are plausible and are to some extent possible to explain. My second hypothesis is, on the face of it, disconfiremd, but we need to factor in the possibility that the models struggle with measuring hybrid regimes. These results are thought provoking, and while not conclusive, are important contributions to a literature that still needs more work. 