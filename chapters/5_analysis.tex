\lettrine{I}{n this chapter I present} the results of the models that have been specified in the previous chapters. I then explain how the results might be interpreted and what what it is we actually see in the results. But before coming to that, I look at some notable examples of the phenomenon in question to get a better understanding of the data at hand. This chapter will also end with a discussion on the validity, reliability, and the robustness of the results, as there are many ways in which to model data, and even small changes may have rather large impacts on the results.

I only find contingent support for my hypothesis one: that linkages to China does influence the level of freedom of expression in a country. The reason for this is that linkages to China have different effects on different types of regimes. I find no evidence for my second hypothesis: that linkages to China will have a bigger impact on hybrid regimes. However, this might come down to the fact that hybrid regimes in general show more variation, and is thus harder to measure. I include several robustness checks to substantiate my conclusions.

\section{Examples}
\begin{table}[H]
\centering
\caption{Changes in freedom of expression and linkages to China}
\label{tab:change}
\begin{tabular}{lLp{15mm}lL}
\toprule
Country & \multicolumn{1}{c}{Freedom} & & Country & \multicolumn{1}{c}{Linkage} \\
\midrule
\cellcolor[HTML]{ff9214} Timor-Leste & 0.776 & & 
\cellcolor[HTML]{ff9214} Cambodia & 0.394 \\
Maldives & 0.636 & & Laos & 0.326 \\
Tunisia & 0.533 & & Turkmenistan & 0.314 \\
Libya & 0.531 & & Malaysia & 0.302 \\
Iraq & 0.510 & & Angola & 0.924 \\
\addlinespace
Hong Kong & -0.568 & & Tunisia & 0.018 \\
Venezuela & -0.670 & & Romania & -0.031 \\
Belarus & -0.735 & & Iran & -0.034 \\
Russia & -0.735 & & The Gambia & -0.067 \\
\cellcolor[HTML]{003F5C}\textcolor{white}{Nicaragua} & -0.869 &  &
\cellcolor[HTML]{003F5C}\textcolor{white}{North Korea} & -0.201 \\
\bottomrule
\multicolumn{5}{p{0.7\textwidth}}{\raggedright{\textit{For freedom score the the difference is measured between 1994 and 2024 \citep{coppedge_v-dem_2025}, while for linkage score it is the difference between 1994 and 2023 \citep{moyer_china-us_2021}.}}}
\end{tabular}
\end{table}

I want to start this section by examining some countries in greater detail to look at how our dependent and independent variables have changed over the time frame we are looking at. For freedom of expression I have data from 1994 to 2024 and for linkages to China I have data from 1994 to 2023. 

Are there any signs that the main hypothesis is true? I have chosen four countries, based on their scores on the two main variables. These are: Timor-Leste, with the largest increase in freedom of expression score; Nicaragua, with the largest decrease in freedom of expression score; Cambodia, with the largest increase in linkages with China; and North Korea, with the largest decrease in linkages with China. The top and bottom five of both variables can be seen in Table \ref{tab:change}.

\begin{figure}[H]
    \centering
    \includegraphics[width=\linewidth]{graphics/single_country_plots.jpeg}
    \caption{Change in linkages to China and freedom of expression (1994-2023/4)}
    \label{fig:scp}
\end{figure}


\subsection{Timor-Leste}
Timor-Leste, also known as East Timor, is a small country in Southeast Asia, located on the east part of the island of Timor, which it shares with its larger neighbour Indonesia. The country has had a fraught history, that reached its climax in 1999, when the country became independent from Indonesia after 24 years of occupation \citep[p. 183]{kingsbury_democratic_2014} In Figure \ref{fig:scp} we can see that there is a large jump in the freedom of expression score in 1999, going from a score as low as 0.06 in 1997 to reaching a high of 0.87 in 2000. After which the score has decreased somewhat, but has generally been stable. While the FBIC index do not include scores before 2002, linkages to China only began rising around 2010, and with a mild increase, but this has not seemed to impact the freedom of expression to any great degree.

Timor-Leste shows that by far the most important factors are domestic, here related to independence. While this is not a boon for my theory, it also does not disconfirm it, as I do not expect linkages to China to be the only factor influencing freedom of expression. The linkage is also not very great, an I would not expect this to have a great impact in any case.

\subsection{Nicaragua}
Nicaragua is a rather poor Central American country, with a turbulent recent history. And recently, since 2006, the country has experienced a rapid democratic backsliding under the leadership of Daniel Ortega. The freedom of expression has seen a particularly steep decline, beating out both Russia and Belarus, however, these countries did have less of a way to fall.

While the linkages to China have slowly increased between 1994 and 2023, in absolute numbers they are quite low, and this again shows the importance of domestic factors when examining the decrease in freedom of expression.

\subsection{Cambodia}
Cambodia is a country on the Indochinese peninsula, which was not spared the ravages of the Cold War, hosting one of the most brutal regimes the world has ever seen in the communist Khmer Rouge. It did subsequently democratise; however, this was never very successful. Freedom of expression has taken a toll in this autocratising process, and in the period between 1994 and 2012, the country recorded a slow but persistent decline. This changed in 2012 as the score declined rapidly, before stabilising at a low level.

It is telling that the repression on freedom of expression began in earnest after a rapid increase in ties to China. The two lines in Figure \ref{fig:scp} seem almost to mirror each other, giving the superficial impression at least, that there is a connection. This is substantiated by the findings of \citet{loughlin_chinese_2021}.

\subsection{North Korea}
The last country I want to look at is North Korea, commonly labelled the hermit kingdom for its super-authoritarian and repressive regime. Its system is founded on self-reliance or Juche, with a strong cult of personality surrounding the ruling Kim clan. North Korea occupies the northern half of the Korean peninsula, and is almost entirely reliant on support from China. Even though Russia has become a closer partner after the 2022 invasion of Ukraine. 

\section{Statistical Significance}
Before proceeding with the analysis, I will make a quick rejoinder on statistical significance. The reason is that I will refer to statistical significance as a convention, however, significance is an arbitrarily chosen threshold; the point where we consider a result to be plausible enough that we reject the null-hypothesis and accept the results as being `true.' The most accepted threshold goes at the five per cent level (\citeauthor{christophersen_introduksjon_2018} \citeyear{christophersen_introduksjon_2018}, pp. 27-31; \citeauthor{gelman_regression_2021} \citeauthor{gelman_regression_2021}, p. 57; \citeauthor{halperin_political_2020} \citeyear{halperin_political_2020},  p. 432; \citeauthor{hellevik_forskningsmetode_2002} \citeyear{hellevik_forskningsmetode_2002}, p. 375; \citeauthor{kellstedt_fundamentals_2018} \citeyear{kellstedt_fundamentals_2018}, pp. 165-166), meaning that if we do the study 100 times, we can expect that in 95 of them of them we end up with a coefficient that is different from zero, and we can plausibly reject the null-hypothesis.

In recent year this has become a hotly debated topic, and rightly so. (++)

\section{Hypothesis One} \label{sec:h1}
In this section and the next, I present the results of the four regression models from the research design chapter. I start of with the simplest versions, before then gradually adding in the control variables, testing the stability of the results. All models use two-way fixed-effects to control for time and country invariables, as this is likely to be a source of unobserved variable bias if not included. Because there is likely to be differences between the countries in the size of the standard errors, I cluster them by country. For readability, I have also decided to highlight the linkages to China variable in the regression tables. It will be orange if there are no significant finds, and blue if there is. This should only be thought of as a reading aid. 

As a quick reminder, hypothesis one states that: \textit{thicker linkages to China will have a negative effect on the level of freedom of expression.} This corresponds to Equation \ref{equ:h1} and \ref{equ:h1_delta}, the difference being that the first model directly regresses linkages to China on freedom of expression, and the second regressing changes in linkages to China on freedom of expression. The results of this model is presented in Table \ref{tab:h1} and \ref{tab:h1_delta}. 

\subsection{Direct relationship between dependent and independent variables}
\begin{table}[H]
\centering
\resizebox{\textwidth}{!}{
\begin{talltblr}[         %% tabularray outer open
label=tab:h1,caption={Standard two-way fixed-effects models},
note{}={x p \num{< 0.1}, * p \num{< 0.05}, ** p \num{< 0.01}, *** p \num{< 0.001}},
]                     %% tabularray outer close
{                     %% tabularray inner open
colspec={Q[]Q[]Q[]Q[]Q[]Q[]Q[]},
column{2,3,4,5,6,7}={}{halign=c,},
column{1}={}{halign=l,},
hline{18}={1,2,3,4,5,6,7}{solid, black, 0.05em},
}                     %% tabularray inner close
\toprule
& \textbf{Model 1.1} & \textbf{Model 1.2} & \textbf{Model 1.3} & \textbf{Model 1.4} & \textbf{Model 1.5} & \textbf{Model 1.6} \\ \midrule %% TinyTableHeader
\SetCell{bg=Orange} Linkages to China &
\SetCell{bg=Orange} -0.081 & 
\SetCell{bg=Orange} -0.093 & 
\SetCell{bg=Orange} -0.069 & 
\SetCell{bg=Orange} -0.067 & 
\SetCell{bg=Orange} -0.111 & 
\SetCell{bg=Orange} -0.058 \\
& (0.132) & (0.134) & (0.127) & (0.128) & (0.130) & (0.111) \\
log(GDP per capita) &  & 0.009 & 0.001 & 0.008 & 0.016 & -0.003 \\
&  & (0.019) & (0.020) & (0.022) & (0.023) & (0.020) \\
Resource rents &  &  & 0.000 & 0.000 & 0.000 & 0.000 \\
&  &  & (0.001) & (0.001) & (0.001) & (0.001) \\
Aid &  &  &  & 0.001 & 0.001 & 0.002** \\
&  &  &  & (0.001) & (0.001) & (0.001) \\
Linkages (West) &  &  &  &  & -0.023* & -0.016x \\
&  &  &  &  & (0.012) & (0.009) \\
Electoral autocracy &  &  &  &  &  & 0.085** \\
&  &  &  &  &  & (0.030) \\
Electoral democracy &  &  &  &  &  & 0.238*** \\
&  &  &  &  &  & (0.034) \\
Liberal democracy &  &  &  &  &  & 0.285*** \\
&  &  &  &  &  & (0.037) \\
Num.Obs. & 5154 & 5150 & 4657 & 4648 & 4648 & 4648 \\
Std.Errors & by: country & by: country & by: country & by: country & by: country & by: country \\
R2 Adj. & 0.874 & 0.874 & 0.882 & 0.883 & 0.883 & 0.907 \\
R2 Within Adj. & 0.001 & 0.001 & 0.000 & 0.003 & 0.008 & 0.208 \\
FE: country & X & X & X & X & X & X \\
FE: year & X & X & X & X & X & X \\
\bottomrule
\end{talltblr}
}
\end{table} 

Table \ref{tab:h1} includes six models estimating the effect of linkages to China on freedom of expression, with Model 216 being equivalent to Equation \ref{equ:h1} above. Model 1.1 regress linkages to China directly on freedom of expression without including any control variables. I then gradually include the five control variables described in Section \ref{control}. The models are all using country and time fixed effects, with the freedom of expression scores being lead by one year to exclude reversed causation. Next I will give a rough explanation of the models I include.

In Model 1.1, I simply regress the independent variable on the dependent. This yields a negative coefficient, but too large a standard error for making anything of this estimate. While the main dataset includes 5,154 complete observations over the dependent and independent variables. In Model 1.2, I add a logarithmically transformed GDP per capita variable to the model, observing that this does not make the results significant. I additionally lose 4 observations. I then proceed to Model 1.3 where I add in rents from natural resources, measured as a percentage of GDP. Here again we see no significant changes to the estimates; however, we lose 493 observations, due in large part to the fact that this variable only has information up until 2021. In Model 1.4 I add a variable measuring the aid a country receives as a percent of GNI. This also has no significant impact on the coefficients, and I lose an additional 9 observations, bringing the number of observations to 4,648. Model 1.5 includes a variable measuring linkages to Western countries. This coefficient is significant and it is interesting that my data shows a negative correlation between linkages to the West and freedom of expression. To the last model, Model 1.6, I add a factor variable dividing the observations into regime types. As regime type is a variable constructed partly from the dependent variable, and the reference category is autocracies, it is unsurprising that each factor is strongly positive and significant.

It can be clearly seen in Table \ref{tab:h1} that linkages to China does not have a direct impact on freedom of expression scores. While the sign of the linkage variable is indeed negative, the standard error is quite large, being more than twice the size of the estimated effect in all models.

Other than this, there are a few more things to note. First, the coefficient for GDP per capita is non-significant, probably indicating that most of the variation in this variable is controlled for by using fixed effects. Second, that aid is significant only when also controlling for regime type. Third, that linkages to the West is negative and statistically significant in Model 1.5, but the significance disappears when controlling for regime type. And fourth, that regime type is always positive and statistically significant. This latter observation is unsurprising as all the regime types are more democratic than the reference regime type which is closed autocracy. I will discuss these findings in the next chapter, however, they should be noted here.

\subsection{Relationship between dependent variable and change in the independent variable}

\begin{table}[!hbt]
\centering
\resizebox{\textwidth}{!}{
\begin{talltblr}[         %% tabularray outer open
label=tab:h1_delta,caption={Models using change in the independent variable},
note{}={x p \num{< 0.1}, * p \num{< 0.05}, ** p \num{< 0.01}, *** p \num{< 0.001}},
]                     %% tabularray outer close
{                     %% tabularray inner open
colspec={Q[]Q[]Q[]Q[]Q[]Q[]Q[]},
column{2,3,4,5,6,7}={}{halign=c,},
column{1}={}{halign=l,},
hline{18}={1,2,3,4,5,6,7}{solid, black, 0.05em},
}                     %% tabularray inner close
\toprule
& Model 1.7 & Model 1.8 & Model 1.9 & Model 1.10 & Model 1.11 & Model 1.12 \\ \midrule %% TinyTableHeader
\SetCell{bg=Orange} Linkages to China & 
\SetCell{bg=Orange} 0.185x & 
\SetCell{bg=Orange} 0.186x & 
\SetCell{bg=Orange} 0.176x & 
\SetCell{bg=Orange} 0.193x & 
\SetCell{bg=Orange} 0.191x & 
\SetCell{bg=Orange} 0.136 \\
& (0.101) & (0.101) & (0.105) & (0.108) & (0.107) & (0.088) \\
log(GDP per capita) &  & 0.006 & -0.001 & 0.006 & 0.012 & -0.005 \\
&  & (0.019) & (0.020) & (0.022) & (0.023) & (0.020) \\
Resource rents &  &  & 0.000 & 0.000 & 0.000 & 0.000 \\
&  &  & (0.001) & (0.001) & (0.001) & (0.001) \\
Aid &  &  &  & 0.001 & 0.001x & 0.002** \\
&  &  &  & (0.001) & (0.001) & (0.001) \\
Linkages (West) &  &  &  &  & -0.020x & -0.015x \\
&  &  &  &  & (0.011) & (0.009) \\
Electoral autocracy &  &  &  &  &  & 0.083** \\
&  &  &  &  &  & (0.030) \\
Electoral democracy &  &  &  &  &  & 0.234*** \\
&  &  &  &  &  & (0.034) \\
Liberal democracy &  &  &  &  &  & 0.277*** \\
&  &  &  &  &  & (0.036) \\
Num.Obs. & 4968 & 4964 & 4478 & 4471 & 4471 & 4471 \\
Std.Errors & by: country & by: country & by: country & by: country & by: country & by: country \\
R2 Adj. & 0.877 & 0.877 & 0.884 & 0.885 & 0.886 & 0.908 \\
R2 Within Adj. & 0.004 & 0.004 & 0.004 & 0.007 & 0.011 & 0.206 \\
FE: country & X & X & X & X & X & X \\
FE: year & X & X & X & X & X & X \\
$\Delta$ Years Linkages & 3 & 3 & 3 & 3 & 3 & 3 \\
\bottomrule
\end{talltblr}
}
\end{table} 

While in Table \ref{tab:h1} I directly regressed the independent variable on the dependent, in Table \ref{tab:h1_delta} I regress the three year change in the independent variable on the dependent variable. I do this because I am interested in looking at the how a change in linkages might affect freedom of expression, not just whether high values on the linkage-variable go together with low values on the freedom of expression-variable. I also consider it likely that the changes would have to take place over an extended period of time, as there is considerable fluctuation from year to year. All the other specifications are the same, where I use two-way fixed-effects and a one year lead on the dependent variable. This model is equivalent to Equation \ref{equ:h2}.

The first model, Model 1.7, regresses change in linkages to China on the freedom of expression score lead by one year. This has a total of 4,968 observations. This configuration, in contrast to directly regressing the linkage and freedom scores on each other, shows a positive relationship between linkages to China and freedom of expression. That is to say that an increase in linkages to China over a three-year period is associated with less repression of freedom of expression. However, this is only significant at the ten per cent level, indicating that this should be interpreted with a great deal of scepticism. Model 1.8 adds the logarithmically transformed GDP per capita variable, with no major changes from Model 1.7, except for losing four observations. In Model 1.9 I add resource rents to the model. This also has barely any impact on the coefficients and we lose 486 observations. To Model 1.10 is added the aid variable, and in the process an additional 7 observations are lost, giving the rest of the models a baseline of 4,471 observations. This also does not cause any significant changes. To Model 1.11 I add the variable controlling for linkages to the West. The coefficient of linkages to China remains positive and significant on the ten per cent level, while the variable controlling for linkages to the West is again negative but significant only at the ten per cent level. The coefficient for aid has also become significant at the same level by this inclusion. The last model in Table \ref{tab:h1_delta} is Model 1.12, which contains the factorised regime variable. The coefficient of linkages to China is no longer significant, but now the aid variable is significant at the one per cent level. The size is, however, rather small. 

The main takeaways of Table \ref{tab:h1_delta} is that the results are fairly stable, not switching signs, and that the effect of linkages to China is not significant. While the results are significant on the ten percent level, this is lower than the commonly accepted threshold of five per cent, and means that one in ten times our model would reject the null-hypothesis even when it was right. This is quite often, and we should not read anything significant into the results, even if we should note the limited plausibility of this result and the implications it has on our theory, which states the opposite of what is here estimated. This result also receives some additional qualification when looking at Table \ref{tab:h2_delta}, where the effects of linkages to China differs based on regime type.

\section{Hypothesis two} \label{sec:h2}
In hypothesis two, I set out to study if the linkage effect is regime dependent. It will be remembered from the last chapter of the theory section that I hypothesis two states that: \textit{Thicker linkages to China will have greater effect on the level of freedom of expression in hybrid regimes}. To assess the truth of this hypothesis, I add an interaction term between the linkages to China variable and the regime variable. The resulting interactions show the effect of linkages on a certain type of regime, with closed autocracies being the reference category. For the purposes of hypothesis two I consider closed autocracies and liberal democracies to be `consolidated regimes,' while the electoral autocracies and democracies are consider hybrid regimes. All the other specifications are similar to hypothesis one. 

\subsection{Standard relationship with an interaction}

\begin{table}[!hbt]
\centering
\resizebox{\textwidth}{!}{
\begin{talltblr}[         %% tabularray outer open
label=tab:h2,caption={Models with interaction between linkages and regime types},
note{}={x p \num{< 0.1}, * p \num{< 0.05}, ** p \num{< 0.01}, *** p \num{< 0.001}},
]                     %% tabularray outer close
{                     %% tabularray inner open
colspec={Q[]Q[]Q[]Q[]Q[]Q[]},
column{2,3,4,5,6}={}{halign=c,},
column{1}={}{halign=l,},
hline{24}={1,2,3,4,5,6}{solid, black, 0.05em},
}                     %% tabularray inner close
\toprule
& Model 2.1 & Model 2.2 & Model 2.3 & Model 2.4 & Model 2.5 \\ \midrule %% TinyTableHeader
\SetCell{bg=Orange} Linkages to China & 
\SetCell{bg=Orange} -0.047 & 
\SetCell{bg=Orange} -0.025 & 
\SetCell{bg=Orange} -0.096 & 
\SetCell{bg=Orange} -0.096 & 
\SetCell{bg=Orange} -0.124 \\
& (0.218) & (0.230) & (0.256) & (0.249) & (0.252) \\
Electoral autocracy & 0.094** & 0.097** & 0.071* & 0.075* & 0.074* \\
& (0.036) & (0.036) & (0.034) & (0.034) & (0.034) \\
Electoral democracy & 0.261*** & 0.265*** & 0.234*** & 0.237*** & 0.235*** \\
& (0.039) & (0.040) & (0.039) & (0.039) & (0.039) \\
Liberal democracy & 0.313*** & 0.318*** & 0.292*** & 0.295*** & 0.288*** \\
& (0.041) & (0.042) & (0.043) & (0.043) & (0.041) \\
\SetCell{bg=Orange} China x El.Aut. & 
\SetCell{bg=Orange} -0.044 & 
\SetCell{bg=Orange} -0.045 & 
\SetCell{bg=Orange} 0.113 & 
\SetCell{bg=Orange} 0.117 & 
\SetCell{bg=Orange} 0.115 \\
& (0.290) & (0.294) & (0.312) & (0.301) & (0.299) \\
\SetCell{bg=Orange} China x El.Dem. & 
\SetCell{bg=Orange} -0.090 & 
\SetCell{bg=Orange} -0.092 & 
\SetCell{bg=Orange} -0.018 & 
\SetCell{bg=Orange} -0.012 & 
\SetCell{bg=Orange} -0.004 \\
& (0.253) & (0.256) & (0.274) & (0.266) & (0.266) \\
\SetCell{bg=Orange} China x Lib.Dem. & 
\SetCell{bg=Orange} -0.238 & 
\SetCell{bg=Orange} -0.268 & 
\SetCell{bg=Orange} -0.207 & 
\SetCell{bg=Orange} -0.207 & 
\SetCell{bg=Orange} -0.126 \\
& (0.260) & (0.277) & (0.291) & (0.280) & (0.280) \\
log(GDP per capita) &  & -0.013 & -0.019 & -0.011 & -0.005 \\
&  & (0.017) & (0.018) & (0.019) & (0.020) \\
Resource rents &  &  & 0.000 & 0.000 & 0.000 \\
&  &  & (0.001) & (0.001) & (0.001) \\
Aid &  &  &  & 0.002** & 0.002** \\
&  &  &  & (0.001) & (0.001) \\
Linkages (West) &  &  &  &  & -0.014 \\
&  &  &  &  & (0.009) \\
Num.Obs. & 5154 & 5150 & 4657 & 4648 & 4648 \\
Std.Errors & by: country & by: country & by: country & by: country & by: country \\
R2 Adj. & 0.902 & 0.901 & 0.905 & 0.907 & 0.907 \\
R2 Within Adj. & 0.216 & 0.217 & 0.201 & 0.207 & 0.209 \\
FE: country & X & X & X & X & X \\
FE: year & X & X & X & X & X \\
\bottomrule
\end{talltblr}
}
\end{table} 

Table \ref{tab:h2} is the same as table Table \ref{tab:h1}, but this time using an interaction term between the linkages to China variable and regime type. This has the unfortunate effect of complicating the model, however, we can now see the effect that linkages have on each regime type. To simplify the reading of the table, we are going to focus most of our attention on the highlighted rows, as these are the coefficients we are mainly interested in. The first highlighted row shows the estimated effect of linkages on China on closed autocracies. Closed autocracies is the reference category, which is why it is estimated by the standard linkage coefficient. In the next highlighted row we find the coefficients for the linkages to China for electoral autocracies, followed by the estimated coefficients for the effect of linkages to China and electoral democracies and liberal democracies respectively.

There are some interesting differences from the models being run in Section \ref{sec:h1}, but non of the estimates for the impact of linkages to China are statistically significant. While it is consistently estimated that linkages to China have a negative effect on freedom of expression for more consolidated regimes, the coefficients for hybrid regimes are less stable, becoming positive for electoral autocracies in Model 2.3. The standard error for all the models are large, being at best the same size as the effect itself, and sometimes much larger. The only significant result is that aid is estimated to be positively and significantly related to freedom of expression, however, the size of the coefficient is very small. 

Considering this result whit respect to the second hypothesis, there is no indication that hybrid regimes are more affected by linkages to China, than are consolidated regimes. In fact, based on this estimation, hybrid regimes looks to be, if anything, less effected. I believe this might come down to the fact that there is much greater variety for hybrid regimes in general, and that this model cannot capture this properly. This is to be discussed later. 

\subsection{Relationship with change in the independent variable and an interaction}

\begin{table}[!hbt]
\centering
\resizebox{\textwidth}{!}{
\begin{talltblr}[         %% tabularray outer open
label=tab:h2_delta,caption={Models with interaction between change in linkages and regime type},
note{}={x p \num{< 0.1}, * p \num{< 0.05}, ** p \num{< 0.01}, *** p \num{< 0.001}},
]                     %% tabularray outer close
{                     %% tabularray inner open
colspec={Q[]Q[]Q[]Q[]Q[]Q[]},
column{2,3,4,5,6}={}{halign=c,},
column{1}={}{halign=l,},
hline{24}={1,2,3,4,5,6}{solid, black, 0.05em},
}                     %% tabularray inner close
\toprule
& Model 2.6 & Model 2.7 & Model 2.8 & Model 2.9 & Model 2.10 \\ \midrule %% TinyTableHeader
\SetCell{bg=Blue} \textcolor{white}{Linkages to China} & 
\SetCell{bg=Blue} \textcolor{white}{0.217x} & 
\SetCell{bg=Blue} \textcolor{white}{0.219x} & 
\SetCell{bg=Blue} \textcolor{white}{0.242*} & 
\SetCell{bg=Blue} \textcolor{white}{0.305*} & 
\SetCell{bg=Blue} \textcolor{white}{0.300*} \\
& (0.111) & (0.111) & (0.120) & (0.134) & (0.134) \\
Electoral autocracy & 0.083** & 0.087** & 0.077* & 0.081** & 0.080** \\
& (0.030) & (0.030) & (0.030) & (0.030) & (0.030) \\
Electoral democracy & 0.246*** & 0.250*** & 0.229*** & 0.233*** & 0.231*** \\
& (0.033) & (0.033) & (0.033) & (0.034) & (0.034) \\
Liberal democracy & 0.289*** & 0.293*** & 0.277*** & 0.281*** & 0.278*** \\
& (0.035) & (0.035) & (0.037) & (0.037) & (0.036) \\
\SetCell{bg=Orange} China x El.Aut. & 
\SetCell{bg=Orange} -0.056 & 
\SetCell{bg=Orange} -0.057 & 
\SetCell{bg=Orange} -0.095 & 
\SetCell{bg=Orange} -0.146 & 
\SetCell{bg=Orange} -0.147 \\
& (0.179) & (0.179) & (0.184) & (0.191) & (0.190) \\
\SetCell{bg=Blue} \textcolor{white}{China x El.Dem.} & 
\SetCell{bg=Blue} \textcolor{white}{-0.294*} & 
\SetCell{bg=Blue} \textcolor{white}{-0.292*} & 
\SetCell{bg=Blue} \textcolor{white}{-0.303*} & 
\SetCell{bg=Blue} \textcolor{white}{-0.380*} & 
\SetCell{bg=Blue} \textcolor{white}{-0.369*} \\
& (0.135) & (0.136) & (0.137) & (0.148) & (0.149) \\
\SetCell{bg=Blue} \textcolor{white}{China x Lib.Dem.} & 
\SetCell{bg=Blue} \textcolor{white}{-0.370**} & 
\SetCell{bg=Blue} \textcolor{white}{-0.373**} & 
\SetCell{bg=Blue} \textcolor{white}{-0.364**} & 
\SetCell{bg=Blue} \textcolor{white}{-0.456**} & 
\SetCell{bg=Blue} \textcolor{white}{-0.406**} \\
& (0.134) & (0.134) & (0.134) & (0.144) & (0.146) \\
log(GDP per capita) &  & -0.013 & -0.018 & -0.009 & -0.005 \\
&  & (0.017) & (0.018) & (0.019) & (0.020) \\
Resource rents &  &  & 0.000 & 0.000 & 0.000 \\
&  &  & (0.001) & (0.001) & (0.001) \\
Aid &  &  &  & 0.002** & 0.002** \\
&  &  &  & (0.001) & (0.001) \\
Linkages (West) &  &  &  &  & -0.012 \\
&  &  &  &  & (0.009) \\
Num.Obs. & 4968 & 4964 & 4478 & 4471 & 4471 \\
Std.Errors & by: country & by: country & by: country & by: country & by: country \\
R2 Adj. & 0.903 & 0.903 & 0.907 & 0.909 & 0.909 \\
R2 Within Adj. & 0.215 & 0.217 & 0.199 & 0.209 & 0.210 \\
FE: country & X & X & X & X & X \\
FE: year & X & X & X & X & X \\
$\Delta$ Years Linkages & 3 & 3 & 3 & 3 & 3 \\
\bottomrule
\end{talltblr}
}
\end{table} 

Table \ref{tab:h2_delta} repeats Table \ref{tab:h1_delta} using a three-year change in linkage score as the independent variable. However, as in the above section, I include an interaction term between linkages to China and regime type. This is by far the most interesting result so far, as we have significant results on the five per cent level for several of our linkages to China coefficients. I show the results in Table \ref{tab:h2_delta}. 

The coefficients of linkages to China, which indicates the effect of linkages to China for a closed autocracy, is positive and significant at the five per cent level for Models 2.8, 2.9, and 2.10. This means that for closed autocracies, an increase in linkages to China increases freedom of expression. This effect seems to be quite strong. Next we have the coefficients for electoral autocracies ,which are not significant and smaller than for closed autocracies, but they are negative. This result is somewhat in line with my first hypothesis, but seems to disprove my second. For electoral democracies, all the models are significant at the five per cent level and strongly negative. This is the same for liberal democracies, where the effect of linkages to China is strongly negative and the coefficients are here significant at the one percent level.

Table \ref{tab:h2_delta}, on the one hand, shows that there is a strong possibility that linkages to China actually has an effect on freedom of expression, but with different effects for each regime type. In this table three out of the four regime types have significant results, with electoral autocracies being the only one without significance. The effect being measured is that an increase in linkages to China over three years, is associated with more freedom of expression in closed autocracies, while linkages to China has the opposite effect on democracies. The estimates for electoral autocracies, while not significant, shows a negative relationship as well, indicating that for electoral autocracies, an increase in linkages to China will reduce freedom of expression. On the other hand, the results of both Table \ref{tab:h2} and \ref{tab:h2_delta} indicates that hypothesis two was wrong. The hybrid regimes are consistently estimated to be less effected by linkages to China than the consolidated regimes. This is a point I will discuss (below)

\subsection{Examples}
To illustrate what I have found in Table \ref{tab:h2_delta}, I will include some examples. I include one example for closed autocracies and one for electoral democracies. First we need to know a bit more about the independent variable. Table \ref{tab:summary} includes some descriptive information of the variable. We see that the mean of change in linkages is about 0, which is not a lot, considering the variable varies from -0.43 to 0.24, and has a standard deviation of 0.04. an increase of one standard deviation in the change score, is expected to lead to a 0.012 point increase in freedom score for closed autocracies, -0.015 decrease for electoral democracies, and a -0.016 decrease for liberal democracies. This is generally down to the fact that for many countries, both the freedom score and linkage score is quite stable, with more dramatic changes occurring only sporadically. 

For one of the more drastic changes, take for instance Nigeria in 2012. In the previous three year period, the country increased its linkage score with China by 0.186, a substantial increase. Since Nigeria is an electoral democracy in 2012, with our model, ceteris paribus, we would expect this to lead to a decrease in freedom of expression score of:
\begin{equation} \label{nigeria_decrease}
    (-0.369 \cdot 0.186)  = -0.069
\end{equation}
giving Nigeria a predicted score of
\begin{equation} \label{nigeria_score}
    0.901 - 0.069 = 0.832
\end{equation}
in 2013. If we compare this against the actual score in 2013, which was 0.893, we see that we overestimate this relationship with a fair amount. This precludes any other factors playing a part, but it is illustrative of the effect size.

We can also take a look at Laos, which is classified as a closed autocracy throughout the time period I am looking at. In 2020 Laos increased its linkages to China with 0.028, a little bit less than a standard deviation in the variable measuring changes in linkages to China. Using this to predict Laos's freedom of democracy score in 2021 we get, assuming that everything else is constant:
\begin{equation}
    0.022 + (0.300 \cdot 0.028) = 0.0304
\end{equation}
This prediction measures up very well with the actual score in 2021 which was 0.030, of course, precluding other changes.

The results from this exercise in predictions highlights two major points. One, that more extreme changes are harder for the model to predict, which is a problem for most models. And two, that the actual size of the effect is not that substantial. It is important to keep in mind, however, that while the year to year effect may not be that substantial, over time it might be a somewhat influential factor in determining where freedom of expression is headed.

\section{Robustness Tests} \label{sec:robust}
To test the robustness of the results, I run several different model specifications to see how well the models hold up to scrutiny. While I present the results here, all the tables referred to in this section can be found in Appendix \ref{apn:robust}. In all I make four changes to my models, to assess how all the models in Tables \ref{tab:h1}, \ref{tab:h1_delta}, \ref{tab:h2}, and \ref{tab:h2_delta} stand up to different forms of scrutiny. First I look at the impact of different time leads on all the models in the four tables presented above. Second I investigate how removing the one year time lead impacts the models. Third I make two tables using models 1.6 and 2.6 to look at how the length of time used for the change in linkages variable impacts the estimation. Finally I replace the linkages to China variable, which heretofore have made use of the FBIC-variable, with the bandwidth variable also featured in the FBIC dataset \citep{moyer_china-us_2021}. Doing this ensures that the estimations are consistent in their results across a wide range of specifications. I find that they are generally consistent, but with some differences. Notable among them are Tables \ref{tab:h2_delta_x_lead} and \ref{tab:h2_bandwidth_delta} which gave significant results.

\subsection{Models with different leads}
I include four additional tables showing different amounts of time lead in Appendix \ref{apn:models}. These are Tables \ref{tab:h1_lead}, \ref{tab:h1_delta_lead}, \ref{tab:h2_lead}, and \ref{tab:h2_delta_lead}. The reason for including these can be seen in Figure \ref{fig:scp}, where Cambodia shows a conspicuous amount of time between an increase in linkages and its effect on freedom of expression. To limit the number of models, I specifically use Models 1.6, 1.12, 2.5, and 2.10, with leads from two, four, six, eight, and ten years. This is to see how large leads affect the result. 

I find that the addition of greater time leads to Model 1.6 increases the size of the coefficients for the linkages to China variable up until eight years, before declining again. For the models with six and eight year time lead, the variable even becomes significant at the ten per cent level. For Model 1.12 the linkage variable seems to generally decrease in size with larger leads. This means that the effect of changes in linkages decays quite quickly, which is to be expected as in many cases they are rather small.

For Model 2.5 we have some strange results. For closed autocracies, the linkage variable suddenly becomes positive at ten years. At the same time,  electoral autocracies, electoral democracies, and liberal democracies, the estimated effect of linkages to China seems to become increasingly negative with longer leads, reaching their lowest point with 10 years of lead. At this point, the estimated coefficients for electoral autocracies and democracies actually becomes significant. I am unable to explain the reason for this, however, the time frame is long enough that we should question this result. (WHY) For Model 2.10 the results are similar to the results of Model 1.12 in that they decrease with longer leads, however, they seem to be at their strongest the first 2-6 years, especially for electoral democracies.

Another interesting point to note is that the variable measuring linkages to the West is negative for all model, as well as being statistically significant in most. This means that more linkages to Western countries goes together with less freedom of expression. I will explore this in the discussion chapter below.

The tables enquiring into the effect of leads show some interesting difference, especially when the ten year lead becomes significant for some coefficients in Table \ref{tab:h2_lead}. However, other than this, the general results confirm our expectations. 

\subsection{Models without lead on the dependent variable}
I next include four more tables running the four main tables of this chapter without lead. I do this to check if the effect might be immediate. I do not expect it to be, however, large discrepancies between models with and without lead should be investigated further. As a caveat to this, it is somewhat more likely that the models using change in linkages to China---instead of the absolute size of them---can be impacted by removing the lead, as they already account for some of the time difference. As I have already noted, the estimated coefficients of the models using change in linkages is larger the smaller the lead, and, in the case of change in the independent variable, the possibility of reversed causation is by definition excluded. This means that we should take very seriously any change in the estimates when using change in linkages as the main independent variable.

Tables \ref{tab:h1_x_lead}, \ref{tab:h1_delta_x_lead}, and \ref{tab:h2_x_lead} shows no significant changes from the models in the analysis chapter. The estimated coefficients are a little different, but nothing that would change any conclusions. Table \ref{tab:h2_delta_x_lead}, however, shows a weaker relationship than does Table \ref{tab:h2_delta} above. While I speculated that, in the case of using change in linkages to China as the dependent variable, removing the lead could make the coefficients larger, the opposite seems to be the case, suggesting that it takes some time for the changes to have an effect.

Removing the lead only have smaller impacts on our estimations. While the results was not consistent with my speculation about how the models running variables with change would be affected, they are nonetheless conforming mostly to expectations.

\subsection{Models with different number of years of change}
The third specification I run, takes its basis in the change in linkages to China. I run five new models with different number of years of change for Models 1.12 and 2.10. The models use variables measuring change varying between one to five years, looking into how this effects the estimates.

Both Table \ref{tab:h1_diff_delta} and \ref{tab:h2_diff_delta} show that the estimates become smaller with changes spanning more years. This effect is strongest for Model 1.12, however it can also be seen in Model 2.10. The latter model is a bit more interesting, however, as we see that the estimates are largest for a two to three year change. The standard errors for a one-year change are also substantive, making the results less certain. This makes sense, as there is much more variation in the changes when the time-span is shorter and the errors are thus likely to be bigger. At a two-year change the errors decrease a lot, and we get significant results.

We thus observe that the different time spans in the change variable is generally consistent with expectations, and that using a three-year change is likely to be a good measure, at least for the model using interaction.

\subsection{Bandwidth}
The last specification I run uses the `bandwidth' variable which is one of the two parts that make up the FBIC index main variable. I include models using this variable to see how removing the importance of the linkages affects the estimations. Bandwidth measures the number of linkages, but not how important these are to the partner of China. According to \citet{levitsky_linkage_2006}, linkages are the main way through which democracy can be spread, with leverage mainly working to strengthen the effects of the linkages. The number of linkages, then, is more important than their size according to this theory. On this basis I include models using bandwidth to check the stability of the results. Again, I rerun all the models in the four main tables of this chapter.

What I find when using bandwidth as the independent variable, is that the models from Table \ref{tab:h1} barely change at all (see Table \ref{tab:h1_bandwidth}). The same is true for Table \ref{tab:h2} (see Table \ref{tab:h2_bandwidth}). However, there are differences when using change in the bandwidth variable. In Table \ref{tab:h1_bandwidth_delta}, the linkages to China variable becomes statistically significant and positive for all models. This is to say that an increase in the number of linkages to China is associated with an increase in the freedom of expression score. On the other hand, in Table \ref{tab:h2_bandwidth_delta}, I find that using change in bandwidth with the regime interaction term reduces the size of the coefficients from that estimated in Table \ref{tab:h2_delta}. Using bandwidth affect the size of the estimates, however, the sign is the same. This is expected as bandwidth leaves out some of the information from the main FBIC variable. 

We can conclude that using the bandwidth variable does not affect the stability of the results in a way that would invalidate them. 

\section{Residuals}
We also need to take a look at residuals to check whether or not there are any problems. I have selected to only focus on the residual plot of Model 2.10, as al the models have residual plots very similar to one another.

A residual plot for a model with a good fit should be narrow with random deviances clustering around zero for every residual. This means that the model predicts values well for all observations, homoscedasticity,  and with no bias.

This is not the case with these models, and here I will explain why this is, what the problem is and how I have proceeded to rectify this.

The residual plot of Model 2.10 is shaped like a diamond around zero. This is to say that the residuals are heteroscedastic, i.e., that the different observations have different prediction errors. While the model is fairly good at predicting observations that have either low or high freedom of expression scores, it struggles with values that are somewhere in the middle. The reason is quite simple, in that there is a upper and lower limit to the dependent variable, and when coming close to this limit the rate of change decreases dramatically. We should also expect there to be some considerable outliers, because of domestic factors like coups or newly gained independence. These are outliers which the model should not be able to predict, without this being a problem. In Figure \ref{fig:residuals} I have highlighted some of the worst predicted observations, which in most cases are countries where large changes have occurred, for instance Afghanistan and the Gambia. 

The good thing about the model is that it is unbiased, thus the coefficients should be estimated correctly. The problem is that the standard errors might be off. To combat this problem I cluster my standard errors by country, which should give me better estimates.

\section{Summary}
To summarise this chapter, I have run several models looking at the relationship between linkages to China and freedom of expression. My main find is that changes in linkages to China has an effect on other countries, however, this effect is dependent on regime type. Closed autocracies seems to increase its freedom of expression when it establishes more linkages to China. The opposite is the case for democracies, where in both electoral and liberal democracies, an increase in linkages to China was associated with reduced freedom of expression. Thus, there is contingent evidence for my first hypothesis. My second hypothesis, however, seems altogether wrong in the light of these results. I will now go on to discuss this more in the next chapter.