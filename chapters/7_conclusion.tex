\lettrine{I}{n this study} I have tried to look into how a large and successful authoritarian regime can impact democracy in other countries by examining China's influence on freedom of expression. Other contributions have found varying degrees of support for China's and other authoritarian countries' influence on democracy, and I have attempted to enquire into one single, important component of democracy, to gain more knowledge about this supposed phenomenon. I dig into a single component because democracy is a many-faceted concept, only measurable by creating complicated indices. This complication may allow for a thorough definition, but may also obscure direct effects of some of the components. To examine this, I decided to look into freedom of expression, as it is the component with the largest decrease in recent time, as measured by the V-Dem institute.

While I find some limited support for my expectation that increased linkages to China have a negative effect on some regime types, the main conclusion is, however, that linkages to China is a minor factor in explaining the variation we see in the freedom of expression scores of countries. Since China is today's most successful authoritarian country, I considered it likely that if any country were to have an effect on freedom of expression it would be China. It is a most likely case, and we can hardly say that it was confirmed, albeit with some small caveats. The size is, whoever, quite small. Putting my results into the context of previous literature on the subject, I consider it more likely that domestic factors are the main contributor, so much so that any international factor is incidental at most. There is at the same time a case for repeating the analysis, not focusing on a single country, but focusing on several of the largest authoritarian in aggregate. While for a single country the impact might be small, the emergence and perseverance of several large authoritarian regimes still holds the door open for an aggregate impact.

While may expectations were only partially correct, this is an important find. It helps close the door on one one of many plausible international contributions to democratic backsliding. While freedom of expression is suffering setbacks, this is far more likely to be caused by domestic, rather than international factors.

The contributions of this study have mainly been empirical, using existing data and concepts to enquire into the effect of international factors on freedom of expression specifically, and democratic backsliding more generally. I am also part of a research agenda focused on China, a rising global super power, which we know surprisingly little about. While the empirical contributions are by far the most important, I have also made some theoretical contribution in refining the theory of how linkages can impact democratic backsliding and freedom of expression. I did this by proposing three mechanisms: learning, immunity, and displacement, which are applicable to China, and any rising and successful power in general.

I will end my thesis by proposing some questions that should be answered by subsequent research. The first is, as mentioned, that studies should look at the aggregated impact of linkages to several autocracies. The sizes of the estimates in this study is comparatively small and it is possible that they may have an effect in the aggregate. Another important step is to examine more countries in detail. In Cambodia there seems to be evidence of this, but why exactly is this relationship so clear, and are there other countries showing the same relationship? This might lead us to discover the conditions under which linkages can have an impact on freedom of expression.

As the freedom and democracy decreases in the world, it is important to understand how this happens. Domestic factors are the most consequential, there is no denying that, but we need to know if and to what extent international factors contribute. This study has advanced this research, corroborating some of the previous findings, and disproving some. 