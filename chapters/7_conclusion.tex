\lettrine{I}{n this study} I have examined how a large and successful authoritarian regime can impact democracy in other countries by examining China's influence on freedom of expression. The academic and public discourses have long considered China to be pushing for autocratisation \citep{jintao_chinas_2023, biden_remarks_2021, economy_exporting_2020, repucci_authoritarians_2022, repucci_global_2022}. Most of the academic contributions have found varying degrees of support for China's and other authoritarian countries' influence on democracy \citep{loughlin_chinese_2021, risse_democracy_2015, tansey_ties_2017, weyland_autocratic_2017, wong_chinese_2019}. Because of the uncertainty of the results, I have attempted to enquire into one single, important component of democracy: freedom of expression. I do this to gain more knowledge about authoritarian diffusion. I examine a single component because the aggregated measure of democracy is complex, possibly obscuring the negative effect of autocratic diffusion. Another reason I have decided to look into freedom of expression, is because it is the component with the largest decrease in recent times, as measured by the V-Dem institute \citep{nord_democracy_2025}.

I find no evidence that linkages to China have any impact on freedom of expression. I run several models, only finding one significant result, albeit one that is very doubtful. Further I also find evidence that one of my proposed mechanisms, immunity, seems to be less likely than what I was expecting. I expected that regime leaders who were inclined to restrict freedom of expression would first establish ties to China to avoid being sanctioned. But my results indicate that Western countries have a negative effect on freedom of expression, possibly meaning that they do not sanction countries when the regime restricts freedom of expression. Since China is today's most successful authoritarian country, I considered it likely that if any country were to have an effect on freedom of expression it would be China. It is a most likely case that was disconfirmed, albeit with the possibility that authoritarian regimes in aggregate may have this effect. My study refutes the claim that China is a major contributor to autocratisation, which is often claimed in both academic and public discourse \citep{jintao_chinas_2023, biden_remarks_2021, economy_exporting_2020, repucci_authoritarians_2022, repucci_global_2022}. I consider it more likely that domestic factors are the main contributor, so much so that any international factor is incidental at most. There is at the same time a case for repeating the analysis, not focusing on links to a single country, but focusing on several of the largest authoritarian countries in the aggregate. While for a single country the impact might be small, the emergence and perseverance of several large authoritarian regimes still might mean that linkages can be the cause of autocratisation.

While my expectations were incorrect, this is an important find. It helps close the door on one of the many plausible international contributions to democratic backsliding. While freedom of expression is suffering setbacks, this is far more likely to be caused by domestic, rather than international factors. Future studies might find evidence of other components being affected by China or by the influence of several authoritarian regimes. 

The contributions of this study have mainly been empirical, using existing data and concepts to enquire into the effect of international factors on freedom of expression specifically, and democratic backsliding more generally. I am a part of a research agenda focused on China, a rising global superpower, which we know surprisingly little about. While the empirical contributions are by far the most important, I have made some theoretical contribution in refining the theory of how linkages can impact democratic backsliding and freedom of expression. I did this by proposing three mechanisms: learning, immunity, and displacement, which are applicable to China, and any rising and successful power in general. I also found that immunity, was less important than expected, which should be taken into consideration by other researchers working in the field. 

I will end my thesis by proposing some questions that should be answered by subsequent research. The first is, as mentioned, that studies should look at the aggregated impact of linkages to several of the major autocracies. The sizes of the estimates in this study is comparatively small, and it is possible that authoritarian regimes only have an effect in the aggregate. Another important step is to examine more countries in detail. In Cambodia there seems to be evidence of this, but why exactly is this relationship so clear, and are there other countries showing the same relationship? This might lead us to discover the conditions under which linkages can have an impact on freedom of expression. This study rules out any general negative, or positive, effect of linkages to China on freedom of expression, but it does not mean that China cannot have this effect on particular countries.

As the freedom and democracy decreases in the world, it is important to understand how this happens. Domestic factors are the most consequential, there is no denying that, but we need to know if and to what extent international factors contribute. This study has advanced this research, corroborating some of the previous findings, and disproving others, hopefully leading to a better understanding of the phenomenon.