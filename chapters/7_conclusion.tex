In this study I have tried to look into how a large and successful authoritarian regime can impact democracy in other countries by examining China's influence on freedom of expression. Other contributions have found varying degrees of support for China's and other authoritarian countries' influence on democracy, and I have attempted to enquire into one single important component of democracy, to gain more knowledge about this supposed phenomenon. I dig into a single component because democracy is a many-faceted concept, only measurable by creating complicated indices. This complication may allow for a thorough definition, but may also obscure direct effects on just some of the indices. I chose to look into freedom of expression as it is the component with the largest decrease in recent time as measured by the V-Dem institute.

While I find some limited support for my expectation that increased linkages to China have a negative effect on some regime types, the main conclusion is, however, that linkages to China can hardly explain the variation we see in the freedom of expression scores of countries. From previous literature on the subject I consider it more likely that domestic factors are the main contributor, so much so that any international factor is incidental at most.

This is an important find, however, since it helps close the door on one of the many research avenues there is on international contributions to democratic backsliding. While freedom of expression is suffering setbacks, this is far more likely to be caused by domestic, rather than international factors. 