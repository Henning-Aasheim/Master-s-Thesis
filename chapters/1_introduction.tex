\lettrine{T}{he year 2025 is already shaping up to be a disaster for democracy} seeing democracy for the average citizen reach a 40 year low \citep[p. 9]{nord_democracy_2025}. The benefits of this reversal all accrues to authoritarian leaders, which faces less constrictions in subjugating their citizens. This is nothing new. Democratic backsliding is observed to happen in waves \citep{huntington_third_1991}, and with the number of democracies at an all time high around 2010 \citep[pp. 10-11]{nord_democracy_2025}, this reversal was almost bound to happen. The wavelike nature of democratisation notwithstanding, democratic backsliding has up until now usually been reserved for new and weaker democracies failing to consolidated. This is not the case this time \citep{nord_democracy_2025}. 

Faced with this situation, we need to understand what is happening. Why are countries re-attaching the yoke of authoritarianism? Countless theories have been proposed, but conclusive evidence has failed to to appear. It is clear that this is a complex issue; we still do not know enough.

Many explanations of this phenomenon are domestic. Economic worries are, as always, a popular explanatory factor (\citeauthor{lipset_chapter_1960} \citeyear{lipset_chapter_1960}; \citeauthor{norris_cultural_2019} \citeyear{norris_cultural_2019}, pp. 132-174). One of the best examples of this is interbellum Germany, where Hitler and the Nazis rose to power ostensibly because Germans were faced with a crushing economic depression, which the moderates were unable to find a solution to. There are also explanations looking at the impact of religion \citep{huntington_third_1991}, generations \citep{norris_cultural_2019}, and immigration \citep[pp. 175-212]{norris_cultural_2019}, among many others.

Domestic factors are likely to be the most important, but in an increasingly interconnected world, other countries might hold the keys to why countries are increasingly turning to authoritarianism. Countries trade, engage in diplomacy, fight, co-operate on security, exchange information like never before. This exchange should effect the participating countries, and opens the door for international factors to affect democracy.

But why should this have a negative effect? The most open countries are Western after all. After the Cold War ended, the West, that is Western Europe, North America, and a handful of other countries, was the pre-eminent force in the world. As time progressed, however, authoritarian states became more visible on the world stage. The prime example is China, which from being close to irrelevant in the 1980s, have soared to become the world's second biggest economy. Russia has also strengthen itself after collapsing in the wake of the dissolution of the Soviet Union,  Saudi Arabia and Iran have become serious players in the Middle East, and Venezuela is still able to command some influence in its home region, despite its economic collapse. The autocratic countries today are stronger than what we have seen in a long time, and this might help explain the wave of autocratisation we see today.

However, there are two opposing effects which are happening at the same time. On the one hand autocratic countries might learn and seek support from authoritarian `black knights' \citeauthor{levitsky_competitive_2010} p. 41; \citeauthor{tolstrup_black_2014} \citeyear{tolstrup_black_2014}, pp.673-674). These are strong authoritarian regime who works to limit democracy. On the other hand, the trade and openness facilitated by having stronger trading partners might serve to strengthen democracy. To effectively study the impact of these black knights, then, we need to disaggregate our measures; open the black box of democracy and peer inside. 

\section{Research Question}
Taking this as a point of departure, I have chosen to look at freedom of expression, a component of democracy that is: one, very important for democracy \citep{dahl_polyarchy_1971, dahl_democracy_1989}; and two, one of the fastest decreasing components of democracy in recent years\citep{nord_democracy_2025}.

I believe that the expansion of trade and other international connections in recent years might have played a part in why freedom of expression is getting more restricted in many countries, prompting me to ask the question:
\begin{displayquote}
    \textit{Can linkages to major powers affect freedom of expression in other countries?}
    \label{rq:general}
\end{displayquote}
This is the overall research question for this thesis, but I do narrow it down a bit further, because I want to look at whether a single country might be able to affect freedom of expression. As my country of choice I have fallen on China. The reason for this is that China today is by far the most prominent authoritarian country. With a population of 1.4 billion people at the end of 2024 \citep{guojia_tongjiju_national_bureau_of_statistics_zong_2024} and a total Gross Domestic Product (GDP) of 19.23 billion dollars\footnote{In current prices} \citep{imf_gdp_2025}. If any authoritarian country should have the ability to affect freedom of expression in another it would be China. I also expect that the effect would be negative, i.e., that freedom of expression decreases with more linkages to China. The modified research question is thus:
\begin{displayquote}
    \textit{Can linkages to China have a negative effect on freedom of expression in other countries?}
    \label{rq:specific}
\end{displayquote}


\section{Thesis Structure}
Having established the programme for the thesis in this introductory section, the challenge for the subsequent sections will be to execute this task in a methodological and easily understandable manner. The next chapter is titled \textit{Literature Review}, which systematically goes through the existing literature. It starts by examining why it is interesting to look at freedom of expression, before connecting this to the larger research field of democratic backsliding and autocratisation. From there I narrow down my research to focus on international causes of this phenomenon, specifically I look at autocratic diffusion and linkages between countries. Linkages will play a particularly important part in my own study so I spend some time giving an overview of this.

The third chapter is \textit{Theory}, wherein I use what I have learned from the literature review to construct a plausible theory for how my dependent and independent variables fit together. I also posit two hypotheses or expectations about what results I will find based on the theory.

Chapter four is \textit{Research Design}, an important chapter where I conceptualise and operationalise my dependent and independent variables. I also give a brief explanation of my control variables, before giving a thorough overview of my research methodology. I use a fixed-effects design, the use of which needs some explaining.

\textit{Analysis} is the fifth chapter, and here I give the results of my regression models. I start the section by looking at some single country examples to gain a better understanding of the phenomenon in question. I then describe the results of my regression models, before looking at the reliability, validity, and substantiality of my results. This is to ensure that the findings are non trivial.

In the penultimate chapter, \textit{Discussion}, I take the results of the analysis and discuss what the results might tell us about China's impact on its partners freedom of expression, and puts this into the larger context of autocratisation studies. This is important because we need to know how my results accords with or modifies previous theories.

The final chapter is the \textit{Conclusion}, where I summarise the most important parts of what I have found and point out some unresolved questions which I think would profit from future research.
