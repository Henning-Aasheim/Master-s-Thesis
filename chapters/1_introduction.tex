\lettrine{T}{he year 2025 is already shaping up to be a disaster for democracy}, seeing democracy for the average citizen reach a 40 year low \citep[p. 9]{nord_democracy_2025}. The benefits of this reversal all accrues to authoritarian leaders, who face fewer restrictions when subjugating their citizens. Democratic backsliding is nothing new, and is observed to happen in waves \citep{huntington_third_1991}. With the number of democracies at an all time high around 2010 \citep[pp. 10-11]{nord_democracy_2025} -- in accordance with the wave-metaphor -- this reversal seemed almost bound to happen. The general idea is that democratic backsliding has usually been reserved for new and weaker democracies, who, being unable to build strong enough democratic institutions, lose out to anti-democratic forces \citep{huntington_third_1991}. But this time something is different. Now institutionalised democracies, notably the USA, seems to be observing autocratisation \citep{nord_democracy_2025}, making it harder to explain the current autocratisation.

Faced with this situation, we need to understand what is happening. Why are countries increasingly turning back to authoritarianism? Countless theories have been proposed, but conclusive evidence has so far failed to appear. This is a complex issue that we need to know more about, but where should we look for answers? 

Most of the proposed explanations of autocratisation revolves around domestic causes. Economic worries is a popular explanatory factor (\citeauthor{lipset_chapter_1960} \citeyear{lipset_chapter_1960}; \citeauthor{norris_cultural_2019} \citeyear{norris_cultural_2019}, pp. 132-174). The example that probably best illustrates this is inter-war Germany. Hitler and the Nazis rose to power ostensibly because Germans were faced with a crushing economic depression, which the moderates were unable to find a solution to. Other contributions have explored the impact of religion \citep{huntington_third_1991}, generational differences \citep{norris_cultural_2019}, and immigration \citep[pp. 175-212]{norris_cultural_2019}. Showcasing just some of the breadth of the research. 

Domestic factors are likely to be the most important, but in an increasingly interconnected world, globalisation might play a key role in explaining autocratisation. Countries trade, engage in diplomacy and security co-operation, fight, and exchange information like never before. Globalisation is thus likely to affect each and every country, and one way this might happen is through authoritarian diffusion \citep{ambrosio_constructing_2010}. Authoritarian diffusion, also called autocratic diffusion, refers to a phenomenon whereby the presence of more and stronger autocracies increase the chance that other countries autocratise as well \citep{ambrosio_constructing_2010}. 

But why should today's globalisation cause countries to become more autocratic? After the Cold War ended, the democratic `West' was the pre-eminent force in the world and this should have lead to democratisation. In later years, however, authoritarian states have become more visible on the world stage. The prime example is China, which went from being a closed off and almost irrelevant country in the 1980s, to become the world's factory and second biggest economy by the 2010s \citep{imf_world_2025}. China might be the exemplar, but other authoritarian countries have also strengthened themselves. After collapsing in the wake of the dissolution of the Soviet Union, Russia has strengthened itself. The same goes for Saudi Arabia and Iran, who have become serious players in the Middle East. In the Americas, Venezuela was able to command some influence in its home region \citep[p. 204]{mcconnell_elite_2024}, at least before the economic collapse from 2013 onwards \citep{imf_world_2025}. The autocratic countries today are stronger than what we have seen in a long time, and this might help explain the wave of autocratisation we see today.

However, based on previous theories, there might be two opposing effects which are happening at the same time. On the one hand autocratic countries might learn and seek support from authoritarian `black knights'\footnote{A `black Knight' is a strong authoritarian (or sometimes democratic) regime that works to guard autocracies and challenge democracies \cite[p. 676]{tolstrup_black_2014}.} (\citeauthor{levitsky_competitive_2010} \citeyear{levitsky_competitive_2010} p. 41; \citeauthor{tolstrup_black_2014} \citeyear{tolstrup_black_2014}, pp.673-674). On the other hand, trade with authoritarian countries might increase prosperity, which can serve to strengthen democracy \citep{lipset_social_1959, przeworski_modernization_1997}. To effectively study the impact of these black knights, then, I need to disaggregate the measures, open the black box of democracy, and peer inside. This is what I propose to do in this study.

\section{Research Question}
Taking the linkages that are facilitated by globalisation as a point of departure, I have decided to look into autocratisation in more detail. Where previous contributions have focused their attention on establishing a connection directly between linkages and autocratic survival \citep{bader_china_2015, tansey_ties_2017}, I propose to enquire into just one single component of democracy. Democracy is hard to define, with scholars differing more or less in their definition \citep{dahl_polyarchy_1971, dahl_democracy_1989, przeworski_democracy_1991, schumpeter_capitalism_2010}. One of the major separations in this literature goes between those who include freedom of expression in their definition of democracy \citep{dahl_polyarchy_1971} and those who do not \citep{przeworski_democracy_1991, schumpeter_capitalism_2010}. I favour the former, as does several of the most used datasets that measures democracy \citep{economist_intelligence_unit_democracy_2024, freedom_house_freedom_2024, marshall_polity5_2020, coppedge_v-dem_2024-1}. Since freedom of expression is one of the fastest decreasing components of democracy in recent years \citep{nord_democracy_2025}, and because it is considered vital to the integrity of democracy \citep{dahl_polyarchy_1971}, I have decided to focus on this component in particular.  

Based on the findings of existing research \citep{ambrosio_constructing_2010, gamso_is_2021, levitsky_linkage_2006, luhrmann_third_2019}, I argue that the expansion of trade and other international connections in recent years might have played a part in why freedom of expression is getting more restricted in many countries, prompting me to ask the question:
\begin{displayquote}
    \textit{To what extent do linkages to major autocratic powers affect freedom of expression?}
    \label{rq:general}
\end{displayquote}
This is the overall research question for this thesis, but I do narrow it down a bit further, because I want to look at whether a single country might be able to affect freedom of expression. As my country of choice, I have decided on China. The reason for this is that China is by far the most prominent authoritarian country. At the end of 2024 it had a population of 1.4 billion people \citep{guojia_tongjiju_national_bureau_of_statistics_zong_2024} and a total Gross Domestic Product (GDP) of 19.23 billion dollars\footnote{In current prices (2025)} \citep{imf_world_2025}, making it the second most populated country in the world, and the second largest economy in 2025. If any authoritarian country should have the ability to affect freedom of expression in other countries it would be China. I expand upon this assumption in the theory section, where I also expect that the effect would be negative, i.e., that freedom of expression decreases with more linkages to China. The modified research question is thus:
\begin{displayquote}
    \textit{To what extent do linkages to China affect freedom of expression?}
    \label{rq:specific}
\end{displayquote}

\section{Findings of the Study}
From my analysis I find that, contrary to my expectations, linkages to China do not affect freedom of expression. I test this with several models, and while the results in most cases are estimated to be negative, they are small and not significant. From this result I conclude that it is very unlikely that linkages to China have any measurable effect on freedom of expression. More surprisingly, I find that linkages to the West are estimated to be negative and significant, albeit the effect is not very sizeable. The effect of linkages to the West is not what I intended to study, so this result should be taken with an extra grain of salt; however, it undermines the mechanism I considered to be most likely to make linkages to China able to influence freedom of expression.

How do my findings impact future research on the subject? First of all, my study largely rules out the impact of linkages to China as a cause of the large decline in freedom of expression. While it is still possible that the linkages can -- in certain situations, most notably Cambodia -- have an impact on freedom of expression, it is not a generally applicable phenomenon. Second, Levitsky and Way's \citeyear{levitsky_linkage_2006} mechanisms of Western democracy promotion comes under question, at least in later years. The West might be more interested in security and stability \citep{borzel_noble_2015, delcour_spoiler_2015, freyburg_local_2015}, making reliance on China less important than expected. Finally, I propose that further research should be more focused on aggregating the effect of several authoritarian countries as the variation in linkages between just one autocracy and the rest of the world has too little variation to be properly measured.

\section{Thesis Structure}
The structure of the thesis is as follows. I start by discussing the the existing literature on freedom of expression, autocratisation, authoritarian diffusion, and linkages in the \textit{Literature Review} chapter. I then used this literature in the \textit{Theory} chapter to build a workable theoretical foundation and create clear expectations for studying the impact of linkages to China on freedom of expression.

In the \textit{Research Design} chapter I explain how I go about creating my models. I include discussions on my methodology and the variables I will control for to make valid inferences. The results of these models are found in the \textit{Analysis} chapter, which starts by going through country examples, before pointing out the most important finds, and finally examining the reliability of the result. In the \textit{Discussion} chapter I place the results of the analysis in the context of the existing literature. I then end the thesis by summarising what I have found and proposing alternative avenues of research to advance our understanding of democratic backsliding.
