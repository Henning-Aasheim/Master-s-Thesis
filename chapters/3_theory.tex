\lettrine{C}{entral to this study} is the mechanism by which China may influence the countries it has linkages with. Diffusion theory teaches us that there are many ways in which the international component can affect the level of, first and foremost, democracy, but also the level of freedom of expression a populace enjoys. To truly understand how our independent variable links to our dependent variable, a more detailed treatment of this relationship is necessary.

To remind the reader, the research question asks whether a country's linkages to China affect the extent of freedom of expression in that country. It is not immediately obvious that they should, but in this chapter, I propose first that they do, and second that the effect of linkages will be moderated by regime type. The following sections explain this in greater detail, but I posit that linkages to China may have a negative impact on freedom of expression in other countries because of a confluence of different factors. I there to be three main reason as to why linkages can impact internal affairs in a country. These are learning, immunity, and displacement. Where learning refers to linkages enabling countries to learn from China. Immunity refers to being supported by China when autocratisation. And lastly, displacement refers to how China is displacing Western democratic countries as a favoured partner, and thus weakening the democratic influence capacity Western countries have on their partner countries.

\section{The Framework}
In this section I build the framework for studying how dyadic linkages might influence freedom of expression. This framework is built up around the three mechanisms mentioned briefly above, and will be expended on in due time. Here I show the rough outline of my theory, before further elaborating on each mechanism in the next three sections. 

I have noted the importance of the contribution of \citet{tansey_ties_2017}, however, it is not a perfect fit. The first problem is that the study concerns authoritarian regime survival and not, as is the case here, with the possibility that an increase in linkages may subsequently increase the authoritarian characteristics of a regime. The second problem is that the \citeauthor{tansey_ties_2017} study is explicitly concerned with ruling out the effect of influential autocratic regimes---so called `black knights' \citep[p.1232]{tansey_ties_2017}. This is the opposite of what this thesis is trying to do. As this is a substantial difference, I will have to make some alterations to the causal mechanisms as proposed in \citet{tansey_ties_2017}. 

The mechanisms of learning and what I have chosen to refer to as immunity---a concept very close to what is termed `decreased salience of autocratic abuse' in Table \ref{tab:linkage}---is going to continue to play a big part in my own theoretical framework. In contrast, the mechanism connected to external stakeholders, in my case China, will stay more in the background. The reason for this is that some contributions find only limited support for this mechanism in regards to China \citep{chen_democracy_2015}. Displacement is a term I use for a phenomenon where linkages to China displaces those of Western countries. The gist of it is that as China becomes more prominent in the international order, it will start to take more space. This space was previously occupied by Western countries, but as time passes, China is taking their place in many different spheres, be they economic, political, or security related. I deem it likely that Western democratic countries engage in some form democracy support \citep{levitsky_linkage_2006}.\footnote{Accepting that this is not always what happens \citep{chen_democracy_2015, wong_chinese_2019}} 

My theory can be summed up by Figure \ref{fig:theory}. First, I take for granted that most countries wish to establish closer linkages to China. In some cases, this might happen because some regimes tries to create a good rapport with an autocratic patron, or, and what is more likely, it happens because it is all but an economic necessity to establish closer linkages to China. This is reflected in Figure \ref{fig:link-china}, that shows how linkages to China has developed in the years between 1994 and 2023. The two dashed boxes in the upper part of Figure \ref{fig:theory} shows the assumption I make that the search for a patron and necessity are the two main reasons why countries would wish to establish linkages at all. However, an in-depth enquiry into the reason for the establishment of linkages is beyond the scope of this study. It should suffice it to say that China is today by far the most powerful autocracy in the world, and its economic power is second only to the USA. 

\begin{figure}
    \centering
    \includegraphics[width=\linewidth]{graphics/chinese_influence.jpeg}
    \caption{Development of linkages to China 1994-2023}
    \label{fig:link-china}
\end{figure}

When linkages are established, the influence of China can start to have an effect. Whether it is a supply or demand relationship is unclear. Authors have attributed it to both \citep[e.g., see:][]{ambrosio_rise_2012, bader_china_2015, brand_authoritarian_2015, economy_exporting_2020, gamso_is_2021, loughlin_chinese_2021, risse_democracy_2015, toettoe_foreign_2023, weyland_autocratic_2017}; however, the literature indicates the latter to be more plausible, with the caveat that China is using the opportunity for what it is worth. I therefore consider linkages, if they have an effect at all, to be a demand-driven mechanism, see Table \ref{tab:mechanism} for more information. 

The next level of Figure \ref{fig:theory} are the three mechanisms: learning, immunity, and displacement, that I have mentioned above. The next sections explains these three mechanisms in greater detail. These mechanisms indicates the most plausible ways in which the linkages to China may cause restrictions on peoples rigth to freely  expression their opinions. 

\begin{figure}[!hbt]
\centering
\resizebox{.8\textwidth}{!}{%
\begin{circuitikz}
\tikzstyle{every node}=[font=\LARGE]
\draw [ rounded corners = 16.2] (5,16) rectangle  node {\LARGE Linkages to China} (12.5,13.5);
\draw [ rounded corners = 16.2] (6.25,12.25) rectangle  node {\LARGE Immunity} (11.25,9.75);
\draw [ rounded corners = 16.2] (0,12.25) rectangle  node {\LARGE Learning} (5,9.75);
\draw [ rounded corners = 16.2] (12.5,12.25) rectangle  node {\LARGE Displacement} (17.5,9.75);
\draw [ rounded corners = 16.2] (3.75,8.5) rectangle  node {\LARGE Decline in freedom of expression} (13.75,6);
\draw [->, >=Stealth] (8.75,13.5) -- (8.75,12.25);
\draw [short] (5,14.75) -- (2.5,14.75);
\draw [->, >=Stealth] (2.5,14.75) -- (2.5,12.25);
\draw [->, >=Stealth] (8.75,9.75) -- (8.75,8.5);
\draw [short] (2.5,9.75) -- (2.5,7.25);
\draw [->, >=Stealth] (2.5,7.25) -- (3.75,7.25);
\draw [->, >=Stealth] (15,14.75) -- (15,12.25);
\draw [ rounded corners = 16.2, dashed] (0,19.75) rectangle  node {\LARGE Patron-seeking}  (7.5,17.25);
\draw [ rounded corners = 16.2, dashed] (10,19.75) rectangle  node {\LARGE Economic necessity}  (17.5,17.25);
\draw [->, >=Stealth, dashed] (6.25,17.25) -- (6.25,16);
\draw [->, >=Stealth, dashed] (11.25,17.25) -- (11.25,16);

\draw [short] (12.5,14.75) -- (15,14.75);
\draw [short] (15,9.75) -- (15,7.25);
\draw [->, >=Stealth] (15,7.25) -- (13.75,7.25);
\end{circuitikz}
}%
\caption{Theory of Chinese influence}
\label{fig:theory}
\end{figure}


\section{Learning}
Learning is in several contributions theorised to be one of the main vehicles of policy diffusion \citep{gilardi_four_2016, shipan_mechanisms_2008, simmons_introduction_2006}; this is not surprising as actors want to strengthen their position by using whatever methods seems to them most effective. And to do this, they look to their peers for guidance. I focus here on learning, as I see decision makers as generally being rational in adopting policies which are either for personal gain or as an end in themselves, i.e., to strengthen their country \citep{shipan_mechanisms_2008}, and believe that it is one of the main vehicles for linkages to China to have an impact on freedom of expression.

What I have simplified in this thesis as the category of `learning', is in the literature quite often divided into two separate mechanisms: learning and imitation/emulation \citep{ elkins_waves_2005, gilardi_four_2016, shipan_mechanisms_2008}. There might not seem to be much difference between them, and I would argue that in most cases there is not; however, in some cases they might be usefully separated. Learning, when used as a separate category can be defined as adopting policies after studying their effects in other places and deeming them the best option. Imitation on the other hand is defined as the adoption of policies based on their earlier adoption by peers and the appropriateness of the policy to the polity \citep[pp. 799-801]{simmons_introduction_2006}. There is much more that can be said about this difference; but, except in noting the use, it will not be relevant to distinguish between these mechanisms, and they have therefore been collapsed into a single category: learning. 

\subsection{Learning and China}

Diplomatic linkages are probably the main way learning can have an effect on the diffusion of policy. \citet[p. 385]{levitsky_linkage_2006} argues that `[l]inkages generates "soft power," or the ability to "shape preferences" and "get others to do what you want." Here diplomatic linkages are one way of achieving this, especially for dominant countries like China and Russia, which smaller countries would look to learn from.

Diplomatic or political linkages show how interconnected two polities are and can include measures like the establishment of embassies and consulates, visits by high-level officials, and participation in international organisations. With repeated interactions in these forums, countries may influence each others as they learn about the policies that the partners have enacted. 

This effect can occur through socialisation, conditionality, and the process of binding \citep[pp. 1323-1326]{ambrosio_catching_2008}. Socialisation refers to transmission of norms and values, which is likely to happen through repeated interactions facilitated by the ties. Conditionality is a more direct way of influence, when one party makes normative demands of a target state. Usually this is done by the major party. The process of binding refers to how ties might bind elites to a certain way of doing things.

When it comes to freedom of expression, these diplomatic ties might even have a more direct effect where, e.g., political ties can serve as a facilitator for teaching how to undermine freedom of expression. China is one of the most advanced countries when it comes to quashing dissenting voices. It has the know-how that other authoritarian leaders would like to get their hands on, and it shares this with other authoritarian regimes \citep[pp. 3-6]{economy_exporting_2020}. For instance, in April 2017, China held a cybersecurity training seminar for Vietnamese officials which was then followed by the introduction of a new Vietnamese cybersecurity law the next year, remarkably similar to the China's own \citep[p. 8]{shahbaz_rise_2018}. 

Some of the learning undoubtedly occurs because countries adopt what from the outside seems like sensible or effective policies from China. However, China is itself a willing teacher (\citeauthor{brazys_chinas_2020} \citeyear{brazys_chinas_2020}, p. 67; \citeauthor{repucci_authoritarians_2022} \citeyear{repucci_authoritarians_2022}, p. 50) and much of it happens through diplomatic linkages. These linkages can take the form of membership of international organisations (\citeauthor{ambrosio_catching_2008} \citeyear{ambrosio_catching_2008}; \citeauthor{economy_exporting_2020} \citeyear{economy_exporting_2020}, pp. 6-7), training of journalists (\citeauthor{brazys_chinas_2020} \citeyear{brazys_chinas_2020}, p. 50; \citeauthor{cook_countering_2022} \citeyear{cook_countering_2022}, pp. 118-119), and the establishment of think tanks \citep[pp. 15-16]{loughlin_chinese_2021} and cultural organisations like the Confucius Institutes \citep{popovic_charm_2020}. 

While the main learning happens in diplomatic forums, it can also happen with economic and security ties; however, I consider these to be strictly secondary. Stronger economic bonds to a partner might serve to socialise countries to the same behaviour; emulating what might seem to be good ideas. With security ties the same socialisation might occur, or the learning might be more direct as when a regime is taught tactics on how to repress freedom of speech. 

Assuming that Beijing is a gravity point in a security network, the closer the co-operation in the security realm, the more a country should learn from China. This learning might then appear as effective repression of freedom of expression. Security forces in countries tied to China might be taught by Chinese instructors and thereby get socialised into a normative environment where repression of freedom of speech is unproblematic. China may also give security forces effective tools for repression. Here it is very likely that they give assistance on technical infrastructure like monitoring and censoring the internet. (READ Hoffman) While both economic and security linkages are likely to have an impact, they would, most likely be moderated through a diplomatic forum, or be so known as not being part of the linkages, but rather norms; a separate issue from what we are concerned with here.

In summary, learning is likely to be a cause of policy diffusion, and, in the case that China has any influence, is a likely route in which linkages may lead to restrictions on freedom of expression.

\section{Immunity}
Immunity is another important reason why countries might limit freedom of expression when they establish linkages with China. My conception of immunity best corresponds to the mechanism that in Figure \ref{tab:linkage} is labelled `decreased salience of autocratic abuse.' However, it also includes attributes from mechanisms two and three, which are labelled external and domestic stakeholders respectively. 

The concept of immunity refers to the fact that countries, regardless of regime, must in some ways work with one another. Now, if a country upsets its major partners, this will undoubtedly have consequences. Immunity, then, is when countries can shirk on certain obligations demanded by some partners, because they have other partners who can substitute for the losses incurred. For a long time western countries were the major partners to almost every country, and this often---but not always \citep{borzel_noble_2015, wong_chinese_2019}---came with strings attached. Western countries usually promote democracy as a part of their support, and this is likely to include demands that countries refrain from repressing freedom of expression. To many leaders of autocratic regimes, this might be a problem, as to secure their power they need to rely on tactics which are diametrically opposed to freedom of expression.

\subsection{Immunity and China}

China is the opposite. China promises a hands-off approach to foreign policy. This is even enshrined in article 4 of the law of foreign relations, stating that:
\begin{displayquote}
\textbf{Article 4} The People's Republic of China pursues an independent foreign policy of peace, and observes the five principles of \textit{mutual respect for sovereignty} and territorial integrity, mutual non-aggression, \textit{mutual non-interference in internal affairs}, equality and mutual benefit, and peaceful coexistence. \citep[emphases are my own]{xinhua_law_2023}
\end{displayquote}
The two key-points here is the mutual respect for sovereignty and mutual non-interference in internal affairs. What this essential means is that China promises not to put any conditions on how a regime operates when engaging with other countries. While the new law is from 2023, this has been China's stance since 1953, when Premier Zhou Enlai first espoused the `Five Principles of Peaceful Co-Existence' in a meeting with Indian officials on the border issues of Tibet \citep{zhonghua_renmin_gongheguo_jiaowenbu_ministry_of_foreign_affairs_of_the_peoples_republic_of_china_zhongguo_2000}.

This fact is very important when looking at how ties to China might affect the level of freedom of expression in other countries. It is likely that elite-level domestic pressures to repress freedom of expression is more easily achieved when a regime faces no external counter-pressures from its international partners. This is immunity from repercussions, and I theorise that it might be an important reason why linkages to China can plausibly effect the level of freedom of expression. Here it is not so much China that is propping up authoritarian regimes, but autocratisation---or authoritarian survival---is an unintended consequence of China establishing linkages to various countries. Here the linkages and the autocratisation occurs because of domestic demand---at least from the regime.

Because of the centrality of economic prosperity to the survival and expansion of any regime, I consider the economic linkages to be the main driver of the immunity mechanism. I consider it probable that most leaders seek to minimise threats to economic performance as this might hurt a regime's popularity.\footnote{It is, however, no clear evidence that a regime's popularity or survival is dependent upon high economic growth \citep{chu_sources_2013, stockemer_economic_2020}. On the other hand, if we consider actors within regimes to be rational and seeking to strengthen their position by increasing their country's material capacity, economic ostracisation should be something to avoid.} In light of this, regimes with a higher density of linkages to Western countries should be less inclined to repress freedom of expression. Inversely, regimes with a high density of linkages to China should be more likely to repress freedom, as this is preferable for domestic reasons and comes without economic costs. 

China's rise to prominence is largely because of its economic gains in the last 30 years, making it almost certain that countries establish ties with it. This might have offered opportunities for authoritarian leaders and demagogues for utilising China as a hedge against economic repercussions from Western partner. But here we face some problems. As China's ascent has made every country more inclined to work with it, this might obscure the actual effect of linkages. The reason is that Western economies are both more democratic and their economies are more open, which would make them strengthen linkages with china, but this would not necessarily cause them to put any restraints on freedom of expression. 

China is also a sizeable donor of aid and loans \citep{fuchs_why_2022}, which is likely to create bonds between elites in China and the receiving country. Regimes leaders get the support they need to keep the regime going, while China is able to offload its over-production and gain favourable concessions from the receiving country. Economically, then, it is important for China to keep the bonds it has established with the ruling elite in a country, and it gives regime leaders have the leeway to crack down on behaviour it objects to and find threatening, like the free exercise of expression.

Other than economic factors, diplomatic and security linkages might be linked to immunity. Diplomatic linkages can have an effect on freedom of expression through China vetoing UN resolutions in the UN Security Council. This might grant countries which flagrantly violate the rights of their citizens more breathing room. China might also give security guarantees to countries, however, I consider this to be unlikely. What is more likely is that China can help countries procure arms if they become ostracised from the west, because they crack down on freedom of expression.

To summarise, linkages to China might immunise countries against the backlash that they would otherwise face if they were beholden to Western countries only. I consider it likely that this works mainly through economic linkages, however, both diplomatic and security linkages may also have an impact.

\section{Displacement}
Displacement refers to the fact that as China's influence grows, China displaces countries' ties with the Western world. Where the West previously held sway, now China reigns supreme. In some cases the displacement is likely to occur because of lingering resentment to former colonisers. The west is seen as hypocritical and bad partners, and since China has become a viable alternative, countries have jumped at the chance to change away from Western partners. Another way in which displacement might happen is that China is willing to support regimes the West is trying to sanction. This is not unlikely to have happened in the case of Venezuela, where strong sanctions on the Maduro regime have decreased Western linkages to Venezuela.

\begin{figure}[hbt!]
\centering
\includegraphics[width = \textwidth]{western_influence.jpeg}
\caption{Change in Western influence}
\label{fig:west}
\end{figure}

Now this is just a speculation on my side, and we need some proof that this is happening. Figure  \ref{fig:west} shows changes in Western\footnote{I have defined the west in this case as any country belonging to the 27 EU countries, as well as all countries with a strong democratic history sins at least the 1990s. The actual countries included in the definition of `The West' can be found in the appendix.} influence for all countries. The orange colour signals a positive change in linkages to the West, while the blue colour signals a negative change. From this we can see that there are some noticeable traces of displacement happening, especially in sub-Saharan Africa. The opposite is true for Europe, where European integration likely has caused Western linkages to strengthen. However there seems to be a trend of China displacing Western ties in some regions of the world, especially sub-Saharan Africa and Central Asia. It would not be unreasonable to think that the change in East-Asia should be far more negative, but the likely reason as to why the West is upholding its position in Southeast Asia is because my definition of the West includes Japan, South Korea, and Taiwan, which are major economic partners to the Southeast Asian countries. 

It is likely that much of the displacement is caused by China's rise as an economic powerhouse. China is the worlds second largest economy, and one of the only great economies that are not Western and democratic. The economic power of China increases China's own heft and norms in the region it displaces Western countries, while at the same time decreasing the power Western countries has to oppose the erosion of freedoms. 

China can also displace Western security linkages. China is the worlds fourth largest arms provider \citep{george_trends_2025, gunter_chinas_2024}, comprising almost six per cent of global exports between 2020 and 2024. However, this is lower than the previous period measured by SIPRI \citep{george_trends_2025}. This decline might down to the fact that China has been rebuilding its military for several years devoting more of its production to domestic use, and Western countries rearming in the face of Russian aggression. The focus on domestic use can perhaps strengthen China as a security partner, a role they are seeking to expand. However, contrary to this, China is a country actively seeking to avoid being entangled in other countries' conflicts, so the only real military alliance China has is with North Korea, showcasing the country's ambivalence to becoming a security guarantor. This makes it doubtful that security linkages are one of the main reasons for displacement, but the fact that China's arms manufacturers have fewer restrictions on whom they sell to \citep{gunter_chinas_2024}, might serve to displace Western influence to some degree.

China is also seeking to displace the West in the realm of diplomacy, to some extent. China has become an important diplomatic partner to many countries, and its position on the UN Security Council makes it Powerful. China has also ambition of creating a more China-centric sphere of influence, sometimes termed the China Model or Beijing Consensus\footnote{The latter as a play on the famous Washington Consensus} \citep{ambrosio_rise_2012, economy_exporting_2020}. This is perhaps most clear in the Shanghai Cooperation Organisation (SCO), which is an international organisation made up of ostensibly authoritarian Central Asian countries, with by far the most powerful country now being China. The organisation is a conservative one, established to preserve the autocratic regimes of the region \citep[p. 1322]{ambrosio_catching_2008}. 

\section{Expectations}
The third wave offers us many cases to study, each of which will be slightly different from all the others. But one of the major changes that has occurred in conjunction with this wave is the rise of China. This is an interesting coincidence, and we should ask the question: are these two phenomena in any way connected?

Previous literature on the subject indicates that we should not expect a concerted effort from China's side to push autocratisation or put pressure on certain freedoms, although this is certainly possible in some cases of particular interest to China, such as Hong Kong \citep{chen_democracy_2015}. We should rather expect to see a gradual change occurring through repeated interaction with China. These interactions are theorised to be measurable through the strength of the linkages between China and other countries.

There is evidence that China really cares about its image abroad and this might influence which countries it gives preference to when it comes to giving aid and preferable loans. However, China also prides itself on its `technocratic' style of government, and we should thus expect China to care more about making the right strategic decisions. These two different goals are quite likely to take place at the same time, and while some countries might get preferential treatment \footnote{There is, however, little evidence of this happening see: \citet{brand_authoritarian_2015}}, we should see a general trend where China forges ties wherever it can.

When China has established the linkages, they begin to have a gradual effect on the partner country's level of freedom of expression. I theorise that there are three main mechanisms through which this effect is channelled: learning, immunity, and favours. the confluence of these factors lead us to the first, and main hypothesis:
\begin{displayquote}
    \textit{H\textsubscript{1}: Thicker linkages to China will have a negative effect on the level of freedom of expression.}  
\end{displayquote}
From previous research \citep{toettoe_foreign_2023}, we know that hybrid regimes are more likely to be affected than other types of regimes. Hybrid, sometimes also known as transitory, regimes are the regimes that show of both democratic and authoritarian characteristics; occupying the centre of the democracy-autocracy spectrum. This gives rise to a secondary hypothesis where hybrid regimes is more likely to be effected by linkages than are consolidated ones. Where electoral autocracies and democracies is likely to be swayed by international pressure, this is less likely to affect closed autocracies and liberal democracies. I reason that this is because the level of freedom of expression is rather set in these societies, and they have robust institutions to handle either attacks or repress freedom of expression. This is not so in hybrid regimes. Hybrid regimes thread a fine line between democracy and autocracy; their regimes usually having a combination of `virtuous' and `perverse' institutions, where democratic and authoritarian characteristics are blended together \citep{valenzuela_democratic_1990}. Hybrid regimes are also more likely to change scores on freedom of expression, because many can be, but are not necessarily, in a transitory phase; either democratising or autocratising.

There is also a ceiling and floor effect at play. on the one hand, liberal democracies cannot increase their level of freedom of expression, because it is almost at the very limit of how unrestricted it can be. On the other hand we find the closed autocracies, where freedom of expression is so low that it cannot reasonably be degraded any further. This leads to secondary hypothesis:
\begin{displayquote}
    \textit{H\textsubscript{2}: Thicker linkages to China will have greater effect on the level of freedom of expression in hybrid regimes.}
\end{displayquote}

These are the three hypotheses I set out to study in the next chapters. The rest of the thesis I dedicate to creating a method of studying these expectations, analysing the results, and discussing what I have found. 